\documentclass[a4paper,12pt]{article}

\usepackage{mystyle}

\title{Typing a $\pi$-calculus with linear logic}
\author{Julien Gabet}
\date{Mars-June, 2018}

\begin{document}

\everymath{\displaystyle}

\maketitle

\section{Elimination of the CUT rule in a simplified system}
We study the following annotated rules for typing simplified $\pi$-calculus with MLL:\\~\\
\indent\textbox{
	Rules for neutral elements:\\~\\
	\begin{prooftree}\infer0{0_x \vdash x:1}\end{prooftree}\hfill
	\begin{prooftree}\hypo{P \vdash \Gamma}
		\infer1{\epsilon_x.P \vdash \Gamma,x:\bot}\end{prooftree}\hfill~\\~\\
	Rules for atoms:\\~\\
	\begin{prooftree}\infer0{A_x \vdash x:a}\end{prooftree}\hfill
	\begin{prooftree}\infer0{x \to y \vdash x:E^\bot,y:E}\end{prooftree}\hfill~\\~\\
	Construction rules:\\~\\
	\begin{prooftree}\hypo{P \vdash \Gamma,x:E}
		\hypo{Q \vdash \Delta,x:F}
		\infer2{P |_x Q \vdash \Gamma,\Delta,x:E \logtensor F}\end{prooftree}\hfill
	\begin{prooftree}\hypo{P \vdash \Gamma,x:E,y:F}
		\infer1{\lambda_xy.P \vdash \Gamma,x:E \logpar F}\end{prooftree}\hfill~
}\\~\\~\\
And give a translation for the left terms:
\begin{align*}
	\lfloor 0_x \rfloor &= 0 \\
	\lfloor x \to y \rfloor &= 0 \\
	\lfloor A_x \rfloor &= A \\
	\lfloor \epsilon_x.P \rfloor &= \lfloor P \rfloor \\
	\lfloor \lambda_xy.P \rfloor &= \lfloor P \rfloor \\
	\lfloor P|_xQ \rfloor &= \lfloor P \rfloor | \lfloor Q \rfloor
\end{align*}
Let's introduce a new CUT rule to cut against terms on the right side:
\[\begin{prooftree}\hypo{P \vdash \Gamma,x:E}
	\hypo{Q \vdash \Delta,x:E^\bot}
	\infer2{P||_xQ \vdash \Gamma,\Delta}\end{prooftree}\]
as well as a translation rule for it:
\[ \lfloor P||_xQ \rfloor = \lfloor P \rfloor | \lfloor Q \rfloor \]
\newpage
\begin{proposition}
This CUT rule is admissible in our system, \ie\\
if $P \vdash \Gamma,x:E$ and $Q \vdash \Delta,x:E^\bot$ then there exists $R$ such that $R \vdash \Gamma,\Delta$\\
with $\lfloor R \rfloor \equiv \lfloor P||_xQ \rfloor$, and the proof of $R$ does not use the CUT rule.
\end{proposition}
\begin{myproof}
	By induction on the couple $(n_1,n_2)$ of the sizes of the proof trees against which the cut is applied:\\
	\underline{If one the last rules is not the one that introduced the variable against which we use the CUT rule},
	without loss of generality, we can assume this rule was the left rule of the cut
	(the other case is a perfect symmetry):\\
	\underline{Case 1:} the last rule is a tensor rule:
	\begin{itemize}
		\item[i:] the variable was introduced in the left subtree of the last rule
		\[
			\begin{prooftree}
						\hypo{(\pi_1)}
					\infer1{P \vdash \Gamma,x:E}
						\hypo{(\pi_2)}
					\infer1{Q \vdash \Delta,x:F,y:G}
				\infer2{P|_xQ \vdash \Gamma,\Delta,x:E\logtensor F,y:G}
					\hypo{(\pi_3)}
				\infer1{R \vdash \Theta,y:G^\bot}
			\infer2[cut]{(P|_xQ)||_yR \vdash \Gamma,\Delta,\Theta,x:E\logtensor F}
			\end{prooftree}
		\]\[
			\leadsto\;\;\;
			\begin{prooftree}
					\hypo{(\pi_1)}
				\infer1{P \vdash \Gamma,x:E}
						\hypo{(\pi_2)}
					\infer1{Q \vdash \Delta,x:F,y:G}
						\hypo{(\pi_3)}
					\infer1{R \vdash \Theta,y:G^\bot}
				\infer2[cut]{Q||_yR \vdash \Delta,\Theta,x:F}
			\infer2{P|_x(Q||_yR) \vdash \Gamma,\Delta,\Theta,x:E\logtensor F}
			\end{prooftree}
		\]
		and we have the structural congruence\\
		$\lfloor (P|_xQ)||_yR \rfloor = (\lfloor P\rfloor|\lfloor Q \rfloor)|\lfloor R \rfloor \equiv \lfloor P \rfloor|(\lfloor Q \rfloor|\lfloor R \rfloor) = \lfloor P|_x(Q||_yR) \rfloor$.
		\item[ii:] the variable was introduced in the right subtree of the last rule
		\[
			\begin{prooftree}
						\hypo{(\pi_1)}
					\infer1{P \vdash \Gamma,x:E,y:G}
						\hypo{(\pi_2)}
					\infer1{Q \vdash \Delta,x:F}
				\infer2{P|_xQ \vdash \Gamma,\Delta,x:E\logtensor F,y:G}
					\hypo{(\pi_3)}
				\infer1{R \vdash \Theta,y:G^\bot}
			\infer2[cut]{(P|_xQ)||_yR \vdash \Gamma,\Delta,\Theta,x:E\logtensor F}
			\end{prooftree}
		\]\[
			\leadsto\;\;\;
			\begin{prooftree}
						\hypo{(\pi_1)}
					\infer1{P \vdash \Gamma,x:E,y:G}
						\hypo{(\pi_3)}
					\infer1{R \vdash \Theta,y:G^\bot}
				\infer2[cut]{P||_yR \vdash \Gamma,\Theta,x:E}
					\hypo{(\pi_2)}
				\infer1{Q \vdash \Delta,x:F}
			\infer2{(P||_yR)|_xQ \vdash \Gamma,\Delta,\Theta,x:E\logtensor F}
			\end{prooftree}
		\]
		and we have\\
		$\lfloor (P|_xQ)||_yR \rfloor = (\lfloor P \rfloor|\lfloor Q \rfloor)|\lfloor R \rfloor \equiv \lfloor P \rfloor|(\lfloor Q \rfloor|\lfloor R \rfloor) \equiv \lfloor P \rfloor|(\lfloor R \rfloor|\lfloor Q \rfloor) \equiv (\lfloor P \rfloor|\lfloor R \rfloor)|\lfloor Q \rfloor = \lfloor (P||_yR)|_xQ \rfloor$.
	\end{itemize}
	\underline{Case 2:} the last rule is not a tensor rule:
	\begin{itemize}
		\item[i:] the last rule is an epsilon rule
		\[
			\begin{prooftree}
						\hypo{(\pi_1)}
					\infer1{P \vdash \Gamma,x:E}
				\infer1{\epsilon_y.P \vdash \Gamma,x:E,y:\bot}
					\hypo{(\pi_2)}
				\infer1{Q \vdash \Delta,x:E^\bot}
			\infer2[cut]{(\epsilon_y.P)||_xQ \vdash \Gamma,\Delta,y:\bot}
			\end{prooftree}
		\]\[
			\leadsto\;\;\;
			\begin{prooftree}
						\hypo{(\pi_1)}
					\infer1{P \vdash \Gamma,x:E}
						\hypo{(\pi_2)}
					\infer1{Q:\Delta,x:E^\bot}
				\infer2[cut]{P||_xQ \vdash \Gamma,\Delta}
			\infer1{\epsilon_y.(P||_xQ) \vdash \Gamma,\Delta,y:\bot}
			\end{prooftree}
		\]
		and we have\\
		$\lfloor (\epsilon_y.P)||_xQ \rfloor = \lfloor P \rfloor | \lfloor Q \rfloor = \lfloor \epsilon_y.(P||_xQ) \rfloor$.
		\item[ii:] the last rule is a lambda rule
		\[
			\begin{prooftree}
						\hypo{(\pi_1)}
					\infer1{P \vdash \Gamma,x:E,y:F,z:G}
				\infer1{\lambda_xy.P \vdash \Gamma,x:E\logpar F,z:G}
					\hypo{(\pi_2)}
				\infer1{Q \vdash \Delta,z:G^\bot}
			\infer2[cut]{(\lambda_xy.P)||_zQ \vdash \Gamma,\Delta,x:E\logpar F}
			\end{prooftree}
		\]\[
			\leadsto\;\;\;
			\begin{prooftree}
						\hypo{(\pi_1)}
					\infer1{P \vdash \Gamma,x:E,y:F,z:G}
						\hypo{(\pi_2)}
					\infer1{Q \vdash \Delta,z:G^\bot}
				\infer2[cut]{P||_zQ \vdash \Gamma,\Delta,x:E,y:F}
			\infer1{\lambda_xy.(P||_zQ) \vdash \Gamma,\Delta,x:E\logpar F}
			\end{prooftree}
		\]
		and we have\\
		$\lfloor (\lambda_xy.P)||_zQ \rfloor = \lfloor P \rfloor | \lfloor Q \rfloor = \lfloor \lambda_xy.(P||_zQ) \rfloor$.
	\end{itemize}
	\underline{If both last rules introduces the variable we cut against}:
	\begin{itemize}
		\item[i:] cut rule for neutrals:
			\[
				\begin{prooftree}
					\infer0{0_x \vdash x:1}
							\hypo{(\pi_1)}
						\infer1{P \vdash \Gamma}
					\infer1{\epsilon_x.P \vdash \Gamma,x:\bot}
				\infer2[cut]{0_x||_x\epsilon_x.P \vdash \Gamma}
				\end{prooftree}
			\]\[
				\leadsto\;\;\;
				\begin{prooftree}
					\hypo{(\pi_1)}
				\infer1{P \vdash \Gamma}
				\end{prooftree}
			\]
			and we have $\lfloor 0_x||_x\epsilon_x.P \rfloor = 0|\lfloor P \rfloor \equiv \lfloor P \rfloor$.
		%\item[ii:] cut rule for atoms:
		%	\[
		%		\begin{prooftree}
		%			\infer0{A_x \vdash x:a}
		%			\infer0{x\to y \vdash x:a^\bot,y:a}
		%		\infer2[cut]{A_x||_xx\to y \vdash y:a}
		%		\end{prooftree}
		%	\]\[
		%		\leadsto\;\;\;
		%		\begin{prooftree}
		%		\infer0{(A_y \equiv) A_x[y/x] \vdash y:a}
		%		\end{prooftree}
		%	\]
		%	and we have $\lfloor A_x||_xx\to y \rfloor = A|0 \equiv A = \lfloor A_x[y/x] \rfloor$.
		%%%% THIS CASE IS TREATED BY THE FOLLOWING ONE %%%%
		\item[ii:] cut rule for axiom:
			\[
				\begin{prooftree}
						\hypo{(\pi)}
					\infer1{P \vdash \Gamma,x:E}
					\infer0{x\to y \vdash x:E^\bot,y:E}
				\infer2[cut]{P||_xx\to y \vdash \Gamma,y:E}
				\end{prooftree}
			\]\[
				\leadsto\;\;\;
				\begin{prooftree}
					\hypo{(\pi[y/x])}
				\infer1{P[y/x] \vdash \Gamma,y:E}
				\end{prooftree}
			\]
			and we have $\lfloor P||_xx\to y \rfloor = \lfloor P \rfloor|0 \equiv \lfloor P \rfloor = \lfloor P[y/x] \rfloor$.\\
			This general form treats all rules that can be cut against an axiom rule.
		\item[iii:] cutting construction rules:
			\[
				\begin{prooftree}
							\hypo{(\pi_1)}
						\infer1{P \vdash \Gamma,x:E}
							\hypo{(\pi_2)}
						\infer1{Q \vdash \Delta,x:F}
					\infer2{P|_xQ \vdash \Gamma,\Delta,x:E\logtensor F}
							\hypo{(\pi_3)}
						\infer1{R \vdash \Theta,x:E^\bot,y:F^\bot}
					\infer1{\lambda_xy.R \vdash \Theta,x:E^\bot\logpar F^\bot}
				\infer2[cut]{(P|_xQ)||_x\lambda_xy.R \vdash \Gamma,\Delta,\Theta}
				\end{prooftree}
			\]\[
				\leadsto\;\;\;
				\begin{prooftree}
							\hypo{(\pi_1)}
						\infer1{P \vdash \Gamma,x:E}
							\hypo{(\pi_3)}
						\infer1{R \vdash \Theta,x:E^\bot,y:F^\bot}
					\infer2[cut]{P||_xR \vdash \Gamma,\Theta,y:F^\bot}
						\hypo{(\pi_2[y/x])}
					\infer1{Q[y/x] \vdash \Delta,y:F}
				\infer2[cut]{(P||_xR)||_yQ[y/x] \vdash \Gamma,\Delta,\Theta}
				\end{prooftree}
			\]
			and we have $\lfloor (P|_xQ)||_x\lambda_xy.R \rfloor = (\lfloor P \rfloor|\lfloor Q \rfloor)|\lfloor R \rfloor \equiv (\lfloor P \rfloor|\lfloor R \rfloor)|\lfloor Q \rfloor = \lfloor (P||_xR)||_yQ[y/x] \rfloor$
	\end{itemize}
With all cases treated (and with the trees marked one can easily verify that the couples of the sizes of the trees decrease each time), this concludes the proof that the cut rule is admissible under our system.
\end{myproof}
\remark The next step is to introduce receiving and sending rules with adequate markings and translations to have them work with the system (and potentially replace the rules for atoms).

\newpage
\section{Sending and receiving names: Roadmap}
Interaction rules:\\~\\
\indent\hfill\begin{prooftree}
		\hypo{P \vdash \Gamma,x:A[v/t]^\bot}
	\infer1{\bar{u}_x\langle v\rangle.P \vdash \Gamma,x:\uparrow_u\exists t.A^\bot}
\end{prooftree}\hfill
\begin{prooftree}
		\hypo{P \vdash \Gamma,x:A\;\;\;t\not\in\Gamma}
	\infer1{u_x(t).P \vdash \Gamma,x\downarrow_u\forall t.A}
\end{prooftree}\hfill
%\begin{prooftree}
%		\hypo{P \vdash \Gamma,x:\uparrow_u E^\bot \logtensor \downarrow_u E}
%	\infer1{(\nu_xu)P \vdash \Gamma}
%\end{prooftree}\hfill
%% NOT FOR NOW -> \nu might not have anything to do on the logic side, which is treated by \mu below.
~\\~\\~\\
Keep arrow rule in atom rules (now a substitution rule).\\
Change cut rule and add a new rule as follow:\\~\\
\indent\hfill\begin{prooftree}
		\hypo{P \vdash \Gamma,x:E}
		\hypo{Q \vdash \Delta,y:E^\bot}
	\infer2{P ||_{(x\leftarrow y)} Q \vdash \Gamma,\Delta}
\end{prooftree}\hfill
\begin{prooftree}
		\hypo{P \vdash \Gamma, x:E \logtensor E^\bot}
	\infer1{\mu_x.P \vdash \Gamma}
\end{prooftree}\hfill~\\~\\
With $\lfloor \mu_x.P \rfloor = \lfloor P \rfloor$.
Eliminate the cut rule throwing it upwards (and introducing an arrow rule, which interpretation is 0).
\ie show that the cut rule holds as before.\\~\\
Prove that: These rules, along with neutral, construction and cut rules, and with usual $\pi$-calculus reduction, keep the typing for reduction.\\
%We might need to annotate the cut rule in cases where it interacts with channels. Also eliminate the $\nu$ rule with this form of cut (transform is immediate when parallel and $\nu$ follow each other).\\~\\~\\
%% NOT NECESSARY WITH THE NEW CUT RULE
Adapt and prove the "confluence"/"congruence" up to permutation on independent names.
\end{document}