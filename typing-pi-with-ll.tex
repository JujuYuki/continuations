\documentclass[a4paper,12pt]{article}

\usepackage{mystyle}

\title{Typing a $\pi$-calculus with linear logic}
\author{Julien Gabet}
\date{Mars-June, 2018}

\begin{document}

\everymath{\displaystyle}

\maketitle

\section{Elimination of the CUT rule in a simplified system}
We study the following annotated rules for typing simplified $\pi$-calculus with MLL:\\~\\
\indent\textbox{
	Rules for neutral elements:\\~\\
	\begin{prooftree}\infer0{0_x \vdash x:1}\end{prooftree}\hfill
	\begin{prooftree}\hypo{P \vdash \Gamma}
		\infer1{\epsilon_x.P \vdash \Gamma,x:\bot}\end{prooftree}\hfill~\\~\\
	Rules for atoms:\\~\\
	\begin{prooftree}\infer0{A_x \vdash x:a}\end{prooftree}\hfill
	\begin{prooftree}\infer0{x \to y \vdash x:E^\bot,y:E}\end{prooftree}\hfill~\\~\\
	Construction rules:\\~\\
	\begin{prooftree}\hypo{P \vdash \Gamma,x:E}
		\hypo{Q \vdash \Delta,x:F}
		\infer2{P |_x Q \vdash \Gamma,\Delta,x:E \logtensor F}\end{prooftree}\hfill
	\begin{prooftree}\hypo{P \vdash \Gamma,x:E,y:F}
		\infer1{\lambda_xy.P \vdash \Gamma,x:E \logpar F}\end{prooftree}\hfill~
}\\~\\~\\
And give a translation for the left terms:
\begin{align*}
	\lfloor 0_x \rfloor &= 0 \\
	\lfloor x \to y \rfloor &= 0 \\
	\lfloor A_x \rfloor &= A \\
	\lfloor \epsilon_x.P \rfloor &= \lfloor P \rfloor \\
	\lfloor \lambda_xy.P \rfloor &= \lfloor P \rfloor \\
	\lfloor P|_xQ \rfloor &= \lfloor P \rfloor | \lfloor Q \rfloor
\end{align*}
\remark the terms we use in our annotated context are terms decorated with fresh names that are completely separate from the names communicated by the corresponding $\pi$-terms, and do not interact with them. As such, we effectively have $\lfloor P\sigma \rfloor = \lfloor P \rfloor$ for all permutation $\sigma$ on the decoration names. That is important as our deduction rules interact over at most one such name, and the remainder of both contexts must be disjoint. This allows for a form of $\alpha$-equivalence: two decorated terms are equivalent if only the names of the free variables differ.\\~\\
Let's introduce a new CUT rule to cut against terms on the right side:
\[\begin{prooftree}\hypo{P \vdash \Gamma,x:E}
	\hypo{Q \vdash \Delta,x:E^\bot}
	\infer2{P||_xQ \vdash \Gamma,\Delta}\end{prooftree}\]
as well as a translation rule for it:
\[ \lfloor P||_xQ \rfloor = \lfloor P \rfloor | \lfloor Q \rfloor \]
\newpage
\begin{proposition}
This CUT rule is admissible in our system, \ie\\
if $P \vdash \Gamma,x:E$ and $Q \vdash \Delta,x:E^\bot$ then there exists $R$ such that $R \vdash \Gamma,\Delta$\\
with $\lfloor R \rfloor \equiv \lfloor P||_xQ \rfloor$, and the proof of $R$ does not use the CUT rule.
\end{proposition}
\begin{myproof}
	By induction on the couple $(n_1,n_2)$ of the sizes of the proof trees against which the cut is applied:\\
	\underline{If one the last rules is not the one that introduced the variable against which we use the CUT rule},
	without loss of generality, we can assume this rule was the left rule of the cut
	(the other case is a perfect symmetry):\\
	\underline{Case 1:} the last rule is a tensor rule:
	\begin{itemize}
		\item[i:] the variable was introduced in the right subtree of the last rule
		\[
			\begin{prooftree}
						\hypo{(\pi_1)}
					\infer1{P \vdash \Gamma,x:E}
						\hypo{(\pi_2)}
					\infer1{Q \vdash \Delta,x:F,y:G}
				\infer2{P|_xQ \vdash \Gamma,\Delta,x:E\logtensor F,y:G}
					\hypo{(\pi_3)}
				\infer1{R \vdash \Theta,y:G^\bot}
			\infer2[cut]{(P|_xQ)||_yR \vdash \Gamma,\Delta,\Theta,x:E\logtensor F}
			\end{prooftree}
		\]\[
			\leadsto\;\;\;
			\begin{prooftree}
					\hypo{(\pi_1)}
				\infer1{P \vdash \Gamma,x:E}
						\hypo{(\pi_2)}
					\infer1{Q \vdash \Delta,x:F,y:G}
						\hypo{(\pi_3)}
					\infer1{R \vdash \Theta,y:G^\bot}
				\infer2[cut]{Q||_yR \vdash \Delta,\Theta,x:F}
			\infer2{P|_x(Q||_yR) \vdash \Gamma,\Delta,\Theta,x:E\logtensor F}
			\end{prooftree}
		\]
		and we have the structural congruence\\
		$\lfloor (P|_xQ)||_yR \rfloor = (\lfloor P\rfloor|\lfloor Q \rfloor)|\lfloor R \rfloor \equiv \lfloor P \rfloor|(\lfloor Q \rfloor|\lfloor R \rfloor) = \lfloor P|_x(Q||_yR) \rfloor$.
		\item[ii:] the variable was introduced in the left subtree of the last rule
		\[
			\begin{prooftree}
						\hypo{(\pi_1)}
					\infer1{P \vdash \Gamma,x:E,y:G}
						\hypo{(\pi_2)}
					\infer1{Q \vdash \Delta,x:F}
				\infer2{P|_xQ \vdash \Gamma,\Delta,x:E\logtensor F,y:G}
					\hypo{(\pi_3)}
				\infer1{R \vdash \Theta,y:G^\bot}
			\infer2[cut]{(P|_xQ)||_yR \vdash \Gamma,\Delta,\Theta,x:E\logtensor F}
			\end{prooftree}
		\]\[
			\leadsto\;\;\;
			\begin{prooftree}
						\hypo{(\pi_1)}
					\infer1{P \vdash \Gamma,x:E,y:G}
						\hypo{(\pi_3)}
					\infer1{R \vdash \Theta,y:G^\bot}
				\infer2[cut]{P||_yR \vdash \Gamma,\Theta,x:E}
					\hypo{(\pi_2)}
				\infer1{Q \vdash \Delta,x:F}
			\infer2{(P||_yR)|_xQ \vdash \Gamma,\Delta,\Theta,x:E\logtensor F}
			\end{prooftree}
		\]
		and we have\\
		$\lfloor (P|_xQ)||_yR \rfloor = (\lfloor P \rfloor|\lfloor Q \rfloor)|\lfloor R \rfloor \equiv \lfloor P \rfloor|(\lfloor Q \rfloor|\lfloor R \rfloor) \equiv \lfloor P \rfloor|(\lfloor R \rfloor|\lfloor Q \rfloor) \equiv (\lfloor P \rfloor|\lfloor R \rfloor)|\lfloor Q \rfloor = \lfloor (P||_yR)|_xQ \rfloor$.
	\end{itemize}
	\underline{Case 2:} the last rule is not a tensor rule:
	\begin{itemize}
		\item[i:] the last rule is an epsilon rule
		\[
			\begin{prooftree}
						\hypo{(\pi_1)}
					\infer1{P \vdash \Gamma,x:E}
				\infer1{\epsilon_y.P \vdash \Gamma,x:E,y:\bot}
					\hypo{(\pi_2)}
				\infer1{Q \vdash \Delta,x:E^\bot}
			\infer2[cut]{(\epsilon_y.P)||_xQ \vdash \Gamma,\Delta,y:\bot}
			\end{prooftree}
		\]\[
			\leadsto\;\;\;
			\begin{prooftree}
						\hypo{(\pi_1)}
					\infer1{P \vdash \Gamma,x:E}
						\hypo{(\pi_2)}
					\infer1{Q:\Delta,x:E^\bot}
				\infer2[cut]{P||_xQ \vdash \Gamma,\Delta}
			\infer1{\epsilon_y.(P||_xQ) \vdash \Gamma,\Delta,y:\bot}
			\end{prooftree}
		\]
		and we have\\
		$\lfloor (\epsilon_y.P)||_xQ \rfloor = \lfloor P \rfloor | \lfloor Q \rfloor = \lfloor \epsilon_y.(P||_xQ) \rfloor$.
		\item[ii:] the last rule is a lambda rule
		\[
			\begin{prooftree}
						\hypo{(\pi_1)}
					\infer1{P \vdash \Gamma,x:E,y:F,z:G}
				\infer1{\lambda_xy.P \vdash \Gamma,x:E\logpar F,z:G}
					\hypo{(\pi_2)}
				\infer1{Q \vdash \Delta,z:G^\bot}
			\infer2[cut]{(\lambda_xy.P)||_zQ \vdash \Gamma,\Delta,x:E\logpar F}
			\end{prooftree}
		\]\[
			\leadsto\;\;\;
			\begin{prooftree}
						\hypo{(\pi_1)}
					\infer1{P \vdash \Gamma,x:E,y:F,z:G}
						\hypo{(\pi_2)}
					\infer1{Q \vdash \Delta,z:G^\bot}
				\infer2[cut]{P||_zQ \vdash \Gamma,\Delta,x:E,y:F}
			\infer1{\lambda_xy.(P||_zQ) \vdash \Gamma,\Delta,x:E\logpar F}
			\end{prooftree}
		\]
		and we have\\
		$\lfloor (\lambda_xy.P)||_zQ \rfloor = \lfloor P \rfloor | \lfloor Q \rfloor = \lfloor \lambda_xy.(P||_zQ) \rfloor$.
	\end{itemize}
	\underline{If both last rules introduces the variable we cut against}:
	\begin{itemize}
		\item[i:] cut rule for neutrals:
			\[
				\begin{prooftree}
					\infer0{0_x \vdash x:1}
							\hypo{(\pi_1)}
						\infer1{P \vdash \Gamma}
					\infer1{\epsilon_x.P \vdash \Gamma,x:\bot}
				\infer2[cut]{0_x||_x\epsilon_x.P \vdash \Gamma}
				\end{prooftree}
			\]\[
				\leadsto\;\;\;
				\begin{prooftree}
					\hypo{(\pi_1)}
				\infer1{P \vdash \Gamma}
				\end{prooftree}
			\]
			and we have $\lfloor 0_x||_x\epsilon_x.P \rfloor = 0|\lfloor P \rfloor \equiv \lfloor P \rfloor$.
		\item[ii:] cut rule for axiom:
			\[
				\begin{prooftree}
						\hypo{(\pi)}
					\infer1{P \vdash \Gamma,x:E}
					\infer0{x\to y \vdash x:E^\bot,y:E}
				\infer2[cut]{P||_xx\to y \vdash \Gamma,y:E}
				\end{prooftree}
			\]\[
				\leadsto\;\;\;
				\begin{prooftree}
					\hypo{(\pi[y/x])}
				\infer1{P[y/x] \vdash \Gamma,y:E}
				\end{prooftree}
			\]
			and we have $\lfloor P||_xx\to y \rfloor = \lfloor P \rfloor|0 \equiv \lfloor P \rfloor = \lfloor P[y/x] \rfloor$.\\
			This general form treats all rules that can be cut against an axiom rule.
		\item[iii:] cutting construction rules:
			\[
				\begin{prooftree}
							\hypo{(\pi_1)}
						\infer1{P \vdash \Gamma,x:E}
							\hypo{(\pi_2)}
						\infer1{Q \vdash \Delta,x:F}
					\infer2{P|_xQ \vdash \Gamma,\Delta,x:E\logtensor F}
							\hypo{(\pi_3)}
						\infer1{R \vdash \Theta,x:E^\bot,y:F^\bot}
					\infer1{\lambda_xy.R \vdash \Theta,x:E^\bot\logpar F^\bot}
				\infer2[cut]{(P|_xQ)||_x\lambda_xy.R \vdash \Gamma,\Delta,\Theta}
				\end{prooftree}
			\]\[
				\leadsto\;\;\;
				\begin{prooftree}
						\hypo{(\pi_1)}
					\infer1{P \vdash \Gamma,x:E}
							\hypo{(\pi_2[y/x])}
						\infer1{Q[y/x] \vdash \Delta,y:F}
							\hypo{(\pi_3)}
						\infer1{R \vdash \Theta,x:E^\bot,y:F^\bot}
					\infer2[cut]{Q[y/x]||_yR \vdash \Delta,\Theta,x:E^\bot}
				\infer2[cut]{P||_x(Q[y/x]||_yR) \vdash \Gamma,\Delta,\Theta}
				\end{prooftree}
			\]
			and we have\\
			$\lfloor (P|_xQ)||_x\lambda_xy.R \rfloor = (\lfloor P \rfloor|\lfloor Q \rfloor)|\lfloor R \rfloor \equiv \lfloor P \rfloor|(\lfloor Q \rfloor|\lfloor R \rfloor) = \lfloor P||_x(Q[y/x]||_yR) \rfloor$.
	\end{itemize}
With all cases treated (and with the trees marked one can easily verify that the couples of the sizes of the trees decrease each time), this concludes the proof that the cut rule is admissible under our system.
\end{myproof}
\remark The next step is to introduce receiving and sending rules with adequate markings and translations to have them work with the system (and potentially replace the rules for atoms).

\newpage
\section{Sending and receiving names}
We add two interaction rules, to replace the atom rule from before:\\~\\
\indent\hfill\begin{prooftree}
		\hypo{P \vdash \Gamma,x:A[v/t]^\bot}
	\infer1{\bar{u}_x\langle v\rangle.P \vdash \Gamma,x:\uparrow_u\exists t.A^\bot}
\end{prooftree}\hfill
\begin{prooftree}
		\hypo{P \vdash \Gamma,x:A\;\;\;t\not\in\Gamma}
	\infer1{u_x(t).P \vdash \Gamma,x:\downarrow_u\forall t.A}
\end{prooftree}\hfill~\\~\\~\\
with $\lfloor \bar{u}_x\langle v\rangle.P \rfloor = \bar{u}\langle v\rangle.\lfloor P\rfloor$ and $\lfloor u_x(t).P \rfloor = u(t).\lfloor P \rfloor$.\\~\\
We need to keep the arrow rule (which acts as a substitution rule), and add a new rule as follows:\\~\\
\indent\hfill\begin{prooftree}
		\hypo{P \vdash \Gamma, x:E \logtensor E^\bot}
	\infer1{\mu_x.P \vdash \Gamma}
\end{prooftree}\hfill with $\lfloor \mu_x.P \rfloor = \lfloor P \rfloor$.\hfill~\\
\begin{proposition}
This CUT rule behaves as a reduction rule, and holds up to observation in our annotated system.
\end{proposition}
\begin{myproof}
Most is already treated in the first system above, here is what's new:\\
If one of the last rules did not introduce the cut variable:
\begin{itemize}
	\item[i:] the last rule is a $\mu$ rule:
		\[
			\begin{prooftree}
						\hypo{(\pi_1)}
					\infer1{P \vdash \Gamma,x:E\logtensor E^\bot,y:F}
				\infer1{\mu_x.P \vdash \Gamma,y:F}
					\hypo{(\pi_2)}
				\infer1{Q \vdash \Delta,y:F^\bot}
			\infer2[cut]{\mu_x.P ||_y Q \vdash \Gamma,\Delta}
			\end{prooftree}
		\]\[
			\leadsto\;\;\;
			\begin{prooftree}
						\hypo{(\pi_1)}
					\infer1{P \vdash \Gamma,x:E\logtensor E^\bot,y:F}
						\hypo{(\pi_2)}
					\infer1{Q \vdash \Delta,y:F^\bot}
				\infer2[cut]{P ||_y Q \vdash \Gamma,\Delta,x:E\logtensor E^\bot}
			\infer1{\mu_x.(P ||_y Q) \vdash \Gamma,\Delta}
			\end{prooftree}
		\]
		Here, $x$ does not appear free in $Q$ in the transformed tree because of the cut rule operating on $y$. This can be achieved by applying a substitution of $x$ for a fresh name on $Q$, and has no importance as substitutions only impact free occurrences of the substituted names in the scope of the term they apply to.\\
		Because of how we project the $\mu$ rule, the congruence holds here.
	\item[ii:] the last rule is a sending rule:
		\[
			\begin{prooftree}
						\hypo{(\pi_1)}
					\infer1{P \vdash \Gamma,x:A[v/t]^\bot,y:E}
				\infer1{\bar{u}_x\langle v\rangle.P \vdash \Gamma,x:\uparrow_u\exists t.A^\bot,y:E}
					\hypo{(\pi_2)}
				\infer1{Q \vdash \Delta,y:E^\bot}
			\infer2[cut]{\bar{u}_x\langle v\rangle.P ||_y Q \vdash \Gamma,\Delta,x:\uparrow_u\exists t.A^\bot}
			\end{prooftree}
		\]\[
			\leadsto\;\;\;
			\begin{prooftree}
						\hypo{(\pi_1)}
					\infer1{P \vdash \Gamma,x:A[v/t]^\bot,y:E}
						\hypo{(\pi_2)}
					\infer1{Q \vdash \Delta,y:E^\bot}
				\infer2[cut]{P||_yQ \vdash \Gamma,\Delta,x:A[v/t]^\bot}
			\infer1{\bar{u}_x\langle v\rangle.(P||_yQ) \vdash \Gamma,\Delta,x:\uparrow_u\exists t.A^\bot}
			\end{prooftree}
		\]
		Here we note that the congruence does not hold for the projection. In the decorated calculus, this transformation is valid because $x$ cannot be free in $Q$, and thus $Q$ cannot interact with the occurrence of $\bar{u}_x$ we observe, meaning the order between the parallel and the sending does not matter to the behavior of the term (this needs some semantics work to clarify). %%TODO: semantics work on why it goes well.
\end{itemize}
If the last rules did both introduce the cut variable (only the interaction case remain):
\[
	\begin{prooftree}
				\hypo{(\pi_1)}
			\infer1{P \vdash \Gamma,x:A\;\;\; t\not\in\Gamma}
		\infer1{u_x(t).P \vdash \Gamma,x:\downarrow_u\forall t.A}
				\hypo{(\pi_2)}
			\infer1{Q \vdash \Delta,x:A[v/t]^\bot}
		\infer1{\bar{u}_x\langle v\rangle.Q \vdash \Delta,x:\uparrow_u\exists t.A^\bot}
	\infer2[cut]{u_x(t).P ||_x \bar{u}_x\langle v\rangle.Q \vdash \Gamma,\Delta}
	\end{prooftree}
\]\[
	\leadsto\;\;\;
	\begin{prooftree}
			\hypo{(\pi_1[v/t])}
		\infer1{P[v/t] \vdash \Gamma,x:A[v/t]}
			\hypo{(\pi_2)}
		\infer1{Q \vdash \Delta,x:A[v/t]^\bot}
	\infer2[cut]{P[v/t] ||_x Q \vdash \Gamma,\Delta}
	\end{prooftree}
\]
Here we observe a reduction behavior. This holds because, as an exterior observer, we cannot tell if an internal action on the term has happened, and this is a basic case of internal action.
\end{myproof}
%%TODO: rules (\nu u):
%%		(\nu u)P | Q \equiv (\nu u)(P | Q), when u \not\in Q
%%		(\nu v)\bar{u}<v>.P or (\nu v)(\bar{u}<v>|P), eg.
%%				(\nu v)(\bar{u}<v>|v.P) that waits for an answer on its private name v
%%					(something that can interact with that is u(t).(\bar{t}|Q), acts as a reception proof)
%%
%%TODO: observationnal equivalence
%%TODO: exponentials, additives.
~\\
The $\nu$ rule should behave as follows:\[
\begin{prooftree}
	\hypo{P \vdash \Gamma\;\;\;\; u \not\in \Gamma}
\infer1{(\nu u)P \vdash \Gamma}
\end{prooftree}\]
where $u$ is a name in the left terms, and should appear merely as a decoration on arrows on the logic part on the right.\\~\\
We want to have usual congruence rules over binding : given that $u$ does not appear free in $Q$, the following holds:
\[\lfloor ((\nu u)P) ||_x Q \rfloor = \lfloor ((\nu u)P) |_x Q \rfloor = (\nu u)\lfloor P \rfloor | \lfloor Q \rfloor \equiv (\nu u)(\lfloor P \rfloor|\lfloor Q \rfloor) = \lfloor (\nu u) (P |_x Q) \rfloor = \lfloor (\nu u) (P ||_x Q) \rfloor\]
~\\
This rule merely serves as a binder on the left terms, and the commutation with the cut rule is immediate (up to renaming of free names in $Q$):
\[\begin{prooftree}
		\hypo{P \vdash \Gamma, x:A\;\;\;\; u\not\in\Gamma,u\not\in A}
	\infer1{(\nu u)P \vdash \Gamma, x:A}
	\hypo{Q \vdash \Delta, x:A^\bot, u\not\in\Delta}
\infer2[cut]{(\nu u)P ||_x Q \vdash \Gamma,\Delta}
\end{prooftree}\]
\[\leadsto\;\;\;\;\begin{prooftree}
		\hypo{P \vdash \Gamma,x:A}
		\hypo{Q \vdash \Delta,x:A^\bot}
	\infer2[cut]{P||_xQ \vdash \Gamma,\Delta}
\infer1{(\nu u)(P ||_x Q)}
\end{prooftree}\]
~\\
Here is an example of good behavior we observe with that: let's imagine some process wants to communicate a private name to some other exterior process, and wait for an answer on that name in order to continue (acting as some sort of delivery report). Such a term in usual, not typed $\pi$-calculus would be $(\nu v)(\bar{u}v | v(t).P)$, and a process able to interact with it would be of the form $u(t).(\bar{t}a|Q)$.\\
Here is what it looks like in our system:\\
The first term is of the form:
\[\begin{prooftree}
				\hypo{x\to y \vdash x:\downarrow_v\forall t'.A, y:\uparrow_v\exists t'.A^\bot}
			\infer1{\bar{u}_x\langle v\rangle.(x\to y) \vdash x:\uparrow_u\exists t.\downarrow_t\forall t'.A, y:\uparrow_v\exists t'.A^\bot}
					\hypo{(\pi_1)}
				\infer1{P \vdash \Gamma, y:A\;\;\; t' \not\in \Gamma}
			\infer1{v_y(t').P \vdash \Gamma, y:\downarrow_v\forall t'.A}
		\infer2[cut]{\bar{u}_x\langle v\rangle.(x\to y) ||_y v_y(t').P \vdash \Gamma, x:\uparrow_u\exists t.\downarrow_t\forall t'.A}
	\infer1{(\nu v)(\bar{u}_x\langle v\rangle.(x\to y) ||_y v_y(t').P) \vdash \Gamma, x:\uparrow_u\exists t.\downarrow_t\forall t'.A}
\end{prooftree}\]
~\\~\\
and the second term of the form:
\[\begin{prooftree}
				\hypo{x\to z \vdash x:A[a/t']^\bot, z:A[a/t']}
			\infer1{\bar{t}_x\langle a\rangle.(x\to z) \vdash x:\uparrow_t\exists t'.A^\bot, z:A[a/t']}
				\hypo{(\pi_2)}
			\infer1{Q \vdash \Delta, z:A[a/t']^\bot}
		\infer2[cut]{\bar{t}_x\langle a\rangle.(x\to z) ||_z Q \vdash \Delta, x:\uparrow_t\exists t'.A^\bot\;\;\; t\not\in \Delta}
	\infer1{u_x(t).(\bar{t}_x\langle a\rangle.(x\to z) ||_z Q) \vdash \Delta, x:\downarrow_u\forall t.\uparrow_t\exists t'.A^\bot}
\end{prooftree}\]
~\\~\\
Then we put them together, and then eliminate the cut with the above depicted transformations:
\[\begin{prooftree}
		\hypo{(\nu v)(\bar{u}_x\langle v\rangle.(x\to y) ||_y v_y(t').P) \vdash \Gamma, x:\uparrow_u\exists t.\downarrow_t\forall t'.A}
		\hypo{u_x(t).(\bar{t}_x\langle a\rangle.(x\to z) ||_z Q) \vdash \Delta, x:\downarrow_u\forall t.\uparrow_t\exists t'.A^\bot}
	\infer2[cut]{(\nu v)(\bar{u}_x\langle v\rangle.(x\to y) ||_y v_y(t').P) ||_x u_x(t).(\bar{t}_x\langle a\rangle.(x\to z) ||_z Q) \vdash \Gamma, \Delta}
\end{prooftree}\]
~\\~\\
Term with annotations of types, full with redundancy:
\begin{align*}
(\nu v)( & \left(\bar{u}_{x:\uparrow_u\exists t.\downarrow_t\forall t'.A}\langle v\rangle.(x^{\downarrow_v\forall t'.A}\to y^{\uparrow_v\exists t'.A^\bot}) ||_y v_{y:\downarrow_v\forall t'.A}(t').P^{\Gamma, y:A}\right)\\
& ||_x u_{x:\downarrow_u\forall t.\uparrow_t\exists t'.A^\bot}(t).\left(\bar{t}_{x:\uparrow_t\exists t'.A^\bot}\langle a\rangle.(x^{A[a/t']^\bot}\to z^{A[a/t']}) ||_z Q^{\Delta,z:A[a/t']^\bot}\right)) \vdash \Gamma,\Delta
\end{align*}
~\\~\\
We only annotate the arrows for the rest of the proof (to reduce weight of annotations, they can be deduced by redrawing the tree from the leaves):\\
\[\leadsto (\nu v)((\bar{u}_x\langle v\rangle.(x^{\downarrow_v\forall t'.A}\to y) ||_y v_y(t').P^\Gamma) ||_x u_x(t).(\bar{t}_x\langle a\rangle.(x^{A[a/t']^\bot}\to z) ||_z Q^\Delta)) \vdash \Gamma,\Delta\]
\[\leadsto (\nu v)((\bar{u}_x\langle v\rangle.(x^{\downarrow_v\forall t'.A}\to y) ||_x u_x(t).(\bar{t}_x\langle a\rangle.(x^{A[a/t']^\bot}\to z) ||_z Q^\Delta)) ||_y v_y(t').P^\Gamma) \vdash \Gamma,\Delta\]
\[\leadsto (\nu v)(((x^{\downarrow_v\forall t'.A}\to y) ||_x (\bar{v}_x\langle a\rangle.(x^{A[a/t']^\bot}\to z) ||_z Q^\Delta)) ||_y v_y(t').P^\Gamma) \vdash \Gamma,\Delta\]
\[\leadsto (\nu v)((((x^{\downarrow_v\forall t'.A}\to y) ||_x \bar{v}_x\langle a\rangle.(x^{A[a/t']^\bot}\to z)) ||_z Q^\Delta) ||_y v_y(t').P^\Gamma) \vdash \Gamma,\Delta\]
\[\leadsto (\nu v)((\bar{v}_y\langle a\rangle.(y^{A[a/t']^\bot}\to z) ||_z Q^\Delta) ||_y v_y(t').P^\Gamma) \vdash \Gamma,\Delta\]
\[\leadsto (\nu v)((\bar{v}_y\langle a\rangle.(y^{A[a/t']^\bot}\to z) ||_y v_y(t').P^\Gamma) ||_z Q^\Delta) \vdash \Gamma,\Delta\]
\[\leadsto (\nu v)((y^{A[a/t']^\bot}\to z ||_y P[a/t']^\Gamma) ||_z Q^\Delta) \vdash \Gamma,\Delta\]
\[\leadsto (\nu v)(P[a/t'][z/y]^{\Gamma,z:A[a/t']} ||_z Q^\Delta) \vdash \Gamma,\Delta\]~\\
And the elimination of cut (especially over $z$) continues on in $P$ and $Q$.
\end{document}