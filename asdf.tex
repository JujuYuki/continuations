\documentclass[a4paper,12pt]{book}

\usepackage{mystyle}

\title{Continuations, mobile processes, all the things...}
\author{Julien Gabet}
\date{Mars-June, 2018}

\begin{document}

\everymath{\displaystyle}

\maketitle

\chapter*{Memo/garbage part}

\noindent$M,N::= x ; \lambda x.M ; MN$\\
$(\lambda x.M)N \rightarrow_\beta M[N/x]$\hfill terms tend to get bigger\\
~\\
If $C[\;]$ is a context, and $M\rightarrow_\beta N$\\
\;\; then $C[M]\rightarrow_\beta C[N]$

\noindent\hrulefill

\noindent$P,Q::= u(xy).P ; \bar{u}xy.P ; P|Q ; (\nu x)P | !P$\\
$u(xy).P | \bar{u}ab.Q \rightarrow P[a/x,b/y] | Q$\\
~\\
If $P\rightarrow Q$ then $C[P]\rightarrow C[Q]$\hfill (with necessary hypothesis on context $C$)\\
If $P\equiv P'\rightarrow Q\equiv Q'$ then $P\rightarrow Q$

\noindent\hrulefill

\noindent Krivine Abstract Machine (KAM)\\
$M\star\Pi\star\EE$\\
\begin{align*}
MN\star\Pi\star\EE &\rightarrow M\star(N,\EE).\Pi\star\EE\\
\lambda x.M\star(N,\EE).\Pi\star\FF &\rightarrow M\star\Pi\star\FF,s\mapsto(N,\EE)\\
x\star\Pi\star\EE,x\mapsto(M,\FF) &\rightarrow M\star\Pi\star\FF
\end{align*}
For exponentials:\\
\indent$!P\simeq!P|!P$\\
\indent$(\nu u)!u(x).P\simeq 0$\\
idea: $!P|Q\simeq!P|!P|Q \;\; \forall Q$

\noindent\hrulefill

\begin{align*}
\llbracket (M,\EE).\Pi \rrbracket_u &= (\nu m)(\nu v)(\bar{u}mv|!m(x)\llbracket M,\EE\rrbracket_x|\llbracket\Pi\rrbracket_v)\\
\llbracket M,\left( x_i\mapsto(M_i,\EE_i)\right)_{i=1..k}\rrbracket_u &= (\nu x_1)\cdots(\nu x_k)(\llbracket M\rrbracket_u|!x_1(u).\llbracket M_1,\EE_1\rrbracket_u | \cdots)\\
\llbracket MN \rrbracket_u &= (\nu v) (\nu n) (\llbracket M \rrbracket_v | \bar{v}nu|!n(x).\llbracket N\rrbracket_x)\\
\llbracket\lambda x.M\rrbracket_u &= u (xv).\llbracket M\rrbracket_v\\
\llbracket x\rrbracket_u &= \bar{x}u
\end{align*}
~\\
We want $M\star\Pi\star\EE\rightarrow M'\star\Pi'\star\EE'$ iff $\llbracket M,\EE\rrbracket_u|\llbracket\Pi\rrbracket_u\rightarrow\llbracket M',\EE'\rrbracket_v|\llbracket\Pi'\rrbracket_v$

~\\

\begin{itemize}
	\item equiv $\simeq$ bisimulation
	\item the traduction goes well
\end{itemize}
\newpage

\begin{definition}
A binary relation $S$ is a reduction bisimulation if, forall $(P,Q)\in S$\\
\begin{itemize}
	\item[(1)] $P\xrightarrow{\tau}P'$ implies $Q\xrightarrow{\tau}Q'$ for some $Q'$ with $(P',Q')\in S$
	\item[(2)] $Q\xrightarrow{\tau}Q'$ implies $P\xrightarrow{\tau}P'$ for some $P'$ with $(P',Q')\in S$
\end{itemize}
\end{definition}


\begin{definition}[Observability:]\\
$P\downarrow_{x}$ if $P$ can make an input action of subject $x$\\
$P\downarrow_{\bar{x}}$ if $P$ can make an output action of subject $x$.
\end{definition}


\begin{definition}[Image-finite process:]\\
$P$ is image-finite if, for all derivative $Q$ of $P$ and any action $\alpha, \exists n\geq0$ and $Q_1,\cdots Q_n$ such that $Q\xRightarrow{\alpha}Q'$ implies $Q'=Q_i$ for some $i$.\\
where $\Rightarrow$ is the reflexive transitive closure of $\xrightarrow{\tau}$ and $\xRightarrow{\alpha}$ is $\Rightarrow\xrightarrow{\alpha}\Rightarrow$ for some action $\alpha$.
\end{definition}

\noindent\hrulefill

\noindent{\Large\underline{Rules for base-$\pi$}}\\
~\\
\indent\textbox{\underline{Value-typing} TV-BASVAL TV-NAME\\
	\begin{flalign*}\text{\underline{Process typing}}	 & \text{T-PAR T-SUM}&\\
		& \text{T-MAT T-NIL}&\\
		& \text{T-REP T-RES T-TAU}&\\
		& \text{T-INP T-OUT}&
	\end{flalign*}}\\
\begin{flalign*}\text{\indent Types: }S,T::= &V \text{ value type}&\\
|&L \text{ link type}&\\
|&\diamond \text{ behaviour type}&
\end{flalign*}
\indent Value types: $V::=B$ basic type\\
\indent Link types: $L::=\sharp V$ connexion type\\
\indent Environments: $\Gamma::=\Gamma,x:L|\Gamma,x:V|\emptyset$
\end{document}