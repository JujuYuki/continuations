%% calculus model
%% normalisation \leadsto and \cong
%% proof confluence
%%%% find two names for annotation variables and actions (channels)
%%%% idea : conclusions VS actions
%%
%% define \equiv as \cong without commuting prefixes
%% define \to as \leadsto without commuting prefixes and acting under prefixes
%%%% Those actually mean remove the non \pi-friendly part
%% See if cut elimination is an emerging property of \to and \equiv in the
%% typed world (it is likely to be the case as it is)

\section{Definitions}

\begin{definition}
Define the calculus model here
\end{definition}

\begin{definition}
Define the $\leadsto$ and $\cong$ relations here
\end{definition}

\begin{proposition}
confluence of $\leadsto$ up to $\cong$
\end{proposition}

\remark that commuting prefixes is not nice looking

\begin{definition}
Define the $\to$ and $\equiv$ restrictions of the two preceding relations here
\end{definition}