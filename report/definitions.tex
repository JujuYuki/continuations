%% calculus model
%% normalisation \leadsto and \cong
%% proof confluence
%%%% find two names for annotation variables and actions (channels)
%%%% idea : conclusions VS actions
%%
%% define \equiv as \cong without commuting prefixes
%% define \to as \leadsto without commuting prefixes and acting under prefixes
%%%% Those actually mean remove the non \pi-friendly part
%% See if cut elimination is an emerging property of \to and \equiv in the
%% typed world (it is likely to be the case as it is)

\section{Definitions}

\begin{definition}
The annotated terms are defined by the following grammar:
\begin{flalign*}P,Q::= & x\to y\;\; ; \;\; 0_x\;\; ; &\\
& \epsilon_x.P\;\; ; \;\; \lambda_xy.P\;\; ; &\\
& u_x(t).P\;\; ; \;\; \bar{u}_x\langle v\rangle.P\;\; ; &\\
& P |_x Q\;\; ; \;\; P ||_x Q \;\; ; &\\
& (\nu u) P. &
\end{flalign*}
\end{definition}

\begin{definition}
We define a cut-eliminating reduction rule as follows:\\
\indent\hfill$\epsilon_x.P||_x 0_x \leadsto P$\hfill $P||_x x\to y \leadsto P[y/x]$\hfill those two rules are symmetrical\\
\indent\hfill$(P|_xQ)||_x \lambda_xy.R \leadsto P||_x (Q[y/x] ||_y R)$\hfill~\hfill~\\
\indent\hfill$\lambda_xy.R ||_x (P|_xQ) \leadsto (R||_xP)||_y Q[y/x]$\hfill~\hfill~\\
\indent\hfill$\bar{u}_x\langle v \rangle.P||_x u_x(t).Q \leadsto P ||_x Q[v/t]$\hfill this rule is symmetrical\\
\indent\hfill$\epsilon_x.P||_yQ \leadsto \epsilon_x.(P||_yQ)$\hfill$\lambda_xy.P||_zQ \leadsto \lambda_xy.(P||_zQ)$\hfill those two are as well\\
\indent\hfill$u_x(t).P||_yQ \leadsto u_x(t).(P||_yQ)$\hfill$\bar{u}_x\langle v \rangle.P||_yQ \leadsto \bar{u}_x\langle v \rangle.(P||_yQ)$\hfill those two are as well\\
\indent\hfill$(P|_xQ)||_yR \leadsto (P||_yR)|_xQ$\hfill if $y$ only appeard in $P$\\
\indent\hfill$(P|_xQ)||_yR \leadsto P|_x(Q||_yR)$\hfill if $y$ only appeard in $Q$\\
\end{definition}

\begin{proposition}
confluence of $\leadsto$ up to $\cong$
\end{proposition}

\remark that commuting prefixes is not nice looking

\begin{definition}
Define the $\to$ and $\equiv$ restrictions of the two preceding relations here
\end{definition}