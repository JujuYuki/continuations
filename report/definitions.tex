\section{The two languages}

\subsection{The classic $\pi$-calculus}

Let us first remind the multiplicative part of the classic $\pi$-calculus and its constructs, as done in~\cite{sangiorgi-2001-pi-calculus}:

\begin{definition}
Given a countable set of names $N:=\{t,u,v,\ldots\}$, the grammar defining the terms of the multiplicative $\pi$-calculus is as follows:\\
\begin{flalign*}
P,Q ::=\;\;	& 0 \;\; ; \;\; P | Q & \text{null action and parallel construction}\\
		& u(t).P \;\; ; \;\; \bar{u}(v).P & \text{sending and receiving names over a channel}\\
		& \tau.P\;\; ; \;\; (\nu u)P & \text{internal action and name binding}
\end{flalign*}
We define the set of free names of a term $P$ by induction on terms as follows:
\begin{align*}
fn(0) &= \emptyset \\
fn(P|Q) &= fn(P)\cup fn(Q)\\
fn(u(t).P) &= (fn(P)\backslash\{t\})\cup\{u\}\\
fn(\bar{u}(v).P) &= fn(P)\cup\{u,v\}\\
fn(\tau.P) &= fn(P)\\
fn((\nu u)P) &= fn(P)\backslash\{u\}
\end{align*}
We also define a structural congruence on terms as follows:
\begin{align*}
P|0 &\equiv P\\
P|Q &\equiv Q|P\\
P|(Q|R) &\equiv (P|Q)|R\\
(\nu u)(\nu v) P &\equiv (\nu v)(\nu u) P\\
(\nu u)0 &\equiv 0\\
(\nu u)(P|Q) &\equiv P|(\nu u)Q\;\;\;\;\text{if }u\not\in fn(P)
\end{align*}
and a reduction relation given by the rules:\\~\\
\indent\hfill\begin{prooftree}\infer0{u(t).P|\bar{u}(v)Q &\to P[v/t]|Q}\end{prooftree}\hfill
\begin{prooftree}\infer0{\tau.P &\to P}\end{prooftree}\hfill~\\~\\
\begin{prooftree}\hypo{P\to P'}\infer1{P|Q\to P'|Q}\end{prooftree}\hfill
\begin{prooftree}\hypo{P\to P'}\infer1{(\nu u)P\to(\nu u)P'}\end{prooftree}\hfill
\begin{prooftree}\hypo{P\equiv Q\to Q'\equiv P'}\infer1{P\to P'}\end{prooftree}\\~\\
where the substitution is defined in an usual fashion by induction on terms.
\end{definition}

The main mechanism here is the sending/receiving system that allows multiple agents to work in parallel and exchange data, this data consisting on names for further communication channels or or even addresses to other processes. This is what modelizes the network system, and this is the core of the non determinism this model allows to study.\\

With this system defined, we would like to look at a few examples:\\

\example a simple process that one could imagine would be something like $\Bigl(u(t).P|Q\Bigr)|\bar{u}(v).R$.\\
This process needs the use of congruence to proceed, and has one reduction:
\begin{align*}
\Bigl(u(t).P|Q\Bigr)|\bar{u}(v).R &\equiv \Bigl(Q|u(t).P\Bigr)|\bar{u}(v).R\\
&\equiv Q|\Bigl(u(t).P|\bar{u}(v).R\Bigr)\\
&\to Q|\Bigl(P[v/t]|R\Bigr)
\end{align*}

\example let's take a term which would want to communicate a private name over a public channel to some other process, in order to use this name as a new channel to communicate with this second process for the rest of execution, especially waiting for a response on this name before continuing execution. Such a term can be of the form $(\nu v)(\bar{u}(v).\tau.0|v(t).P)$. It could then interact with anything of the form $u(t).(\bar{t}(w).\tau.0|Q)$.\\
Putting these together:
\begin{align*}
(\nu v)\Bigl(\bar{u}(v).\tau.0|v(t').P\Bigr)|u(t).\Bigl(\bar{t}(w).\tau.0|Q\Bigr) &\equiv (\nu v)\biggl(\Bigl(\bar{u}(v).\tau.0|v(t).P\Bigr)|u(t).\Bigl(\bar{t}(w).\tau.0|Q\Bigr)\biggr)\\
&\equiv (\nu v)\biggl(\Bigl(v(t').P|\bar{u}(v).\tau.0\Bigr)|u(t).\Bigl(\bar{t}(w).\tau.0|Q\Bigr)\biggr)\\
&\equiv (\nu v)\biggl(v(t').P|\Bigl(\bar{u}(v).\tau.0|u(t).\bigl(\bar{t}(w).\tau.0|Q\bigr)\Bigr)\biggr)\\
&\to (\nu v)\biggl(v(t').P|\Bigl(\tau.0|\bigl(\bar{v}(w).\tau.0|Q[v/t]\bigr)\Bigr)\biggr)\\
&\to (\nu v)\biggl(v(t').P|\Bigl(0|\bigl(\bar{v}(w).\tau.0|Q[v/t]\bigr)\Bigr)\biggr)\\
&\equiv (\nu v)\biggl(v(t').P||\Bigl(\bar{v}(w).\tau.0|Q[v/t]\Bigr)\biggr)\\
&\equiv (\nu v)\biggl(\Bigl(v(t').P|\bar{v}(w).\tau.0\Bigr)|Q[v/t]\biggr)\\
&\to (\nu v)\biggl(\Bigl(P[w/t']|\tau.0\Bigr)|Q[v/t]\biggr)\\
&\to (\nu v)\biggl(\Bigl(P[w/t']|0\Bigr)|Q[v/t]\biggr)\\
&\equiv (\nu v)\Bigl(P[w/t']|Q[v/t]\Bigr)
\end{align*}

We will follow this example in the paper, as it will be able to show a great variety of propositions applied to a practical example.

\remark we see that the reduction $\to$ here is not confluent. A good example for that would be the term\\
$(u(t).P|\bar{u}(v).Q)|u(t).S \equiv u(t).P|(\bar{u}(v).Q|u(t).S)$ that can reduce to either $(P[v/t]|Q)|u(t).S$ or $u(t).P|(Q|S[v/t])$, and those reduction paths do not have a common reduction in general, especially if $P,Q$ and $S$ are independent. That gives us a system that is highly non-deterministic when studied up to structural congruence. That could lead to terms with multiple reduction paths, but some of them could end up blocked at some point while others could end the reduction to the base term 0. This specific behavior with multiple reduction paths possibly being blocked is what we will try to avoid with the decorated calculus, by forcing a reduction path with annotations.

\subsection{The annotated $\pi$-calculus and its projection}

We now need to define the language of terms of our annotated $\pi$-calculus. We introduce the annotation with variables in order to identify what part of a term has to interact with what other part.
\begin{definition}
Take two countable sets, $N:=\{t,u,v,\ldots\}$ the set of names as in the classic $\pi$-calculus, and $V:=\{x,y,z,\ldots\}$ the set of variables for the decorations.\\
The annotated terms are defined by the following grammar:
\begin{flalign*}P,Q::= & x\tto y\;\; ; \;\; 0_x\;\; ; &x,y\in V\;\;\;\text{base terms: equalizer and null action}&\\
& \epsilon_x.P\;\; ; & x\not\in fv(P)\;\;\;\text{variable introduction scheduling prefix}&\\
& \lambda_xy.P\;\; ; & x,y\in fv(P)\;\;\;\text{variables binding scheduling prefix}&\\
& u_x(t).P\;\; ; \;\; \bar{u}_x\langle v\rangle.P\;\; ; &t,u,v\in N,x\in fv(P)\;\;\;\text{action prefixes}&\\
& P |_x Q\;\; ; \;\; P ||_x Q \;\; ; &x\in fv(P)\cap fv(Q)\;\;\;\text{parallel and synchronization}&\\
& (\nu u) P &u\in N\;\;\;\text{name binding prefix}&
\end{flalign*}
where construction $x \tto y$ is commutative, and $fv(P)$ and $fn(P)$ are defined by induction on terms as follows:\\
\begin{align*}
fv(x\tto y) &= \{x,y\} & fn(x\tto y) & = \emptyset\\
fv(0_x) &= \{x\} & fn(0_x) &= \emptyset\\
fv(\epsilon_x.P) &= fv(P)\cup\{x\} & fn(\epsilon_x.P) &= fn(P)\\
fv(\lambda_xy.P) &= fv(P)\backslash\{y\} & fn(\lambda_xy.P) &= fn(P)\\
fv(u_x(t).P) &= fv(P) & fn(u_x(t).P) &= (fn(P)\backslash\{t\})\cup\{u\}\\
fv(\bar{u}_x\langle v\rangle.P) &= fv(P) & fn(\bar{u}_x\langle v\rangle.P) &= fn(P)\cup\{u,v\}\\
fv((\nu u) P) &= fv(P) & fn((\nu u) P) &= fn(P)\backslash\{u\}\\
fv(P |_x Q) &= fv(P)\cup fv(Q) & fn(P |_x Q) &= fn(P)\cup fn(Q)\\
fv(P ||_x Q) &= (fv(P)\cup fv(Q))\backslash\{x\} & fn(P ||_x Q) &= fn(P)\cup fn(Q)\\
\end{align*}
\end{definition}

\remark the only construction with a prefix that adds to the variables of a term is the variable introduction ($\epsilon_x$ construction). That is because, as the rules are defined here for the other prefixes, the variable used to decorate the prefix already exists in the subterm, and the variable is used as a follow-up for the behavior of this term. The $\lambda$ is also binding for its second variable ($y$ here), keeping availability for its first variable only. The $\epsilon$, on the opposite, creates its variable, that does not exist in the base term.

\remark in some constructions and some cases, it can be useful to have a notation to describe in which subterm a variable appears. We will choose for that to put the variable in exponent, so that $(P^x |_y Q) |_x R^x$ denotes the fact that $x$ appears in $P$ and $R$, but not in $Q$.

\remark we also define substitution of names and variables by induction on the terms the exact same way as usual. Though, it is important to note that, because names and variables are two distinct and independent sets, substituting a variable for a name (or the opposite) is not possible. The distinction between those two sets is important for the reduction to pose no problem of variable capture, and allows for something akin to $\alpha$-conversion from $\lambda$-calculi: one can rename freely all instances of a free variable or name in a term for a fresh one, or all instances of a bound variable or name (including the binding operator) for a fresh one as well \emph{in the scope of the binding operator}, and the term is still considered the same.\\

This calculus has very similar constructions to the classic $\pi$-calculus defined above. Of note though is that the parallel operation, the internal action and the null process have two equivalent each in the decorated calculus. For the parallel operation, the meaning is that it puts two processes together, but they might not interact with each other, which is why the variable used to observe them together is kept available to the outside world, while the synchronization puts the processes in interaction with each other on this variable, which is why it is hidden from the exterior, acting as a binding operation. The $\epsilon$ is an internal action that creates a variable to observe on, while the $\lambda$ is an internal action that puts two variables together to observe as one. The $0$ is still an inactive process, which we do not need to observe on more than one variable, and the equalizer is a bit more complex, as it does nothing observable but create two observation variables. This specificity will be addressed in the projection:

\begin{definition}
We define a projection operator $\lfloor\cdot\rfloor$ from the annotated calculus to the classic $\pi$-calculus as follows:\\
\begin{align*}
\lfloor 0_x \rfloor &= 0 \\
\lfloor x \tto y \rfloor &= \tau.0 \\
\lfloor \lambda_xy.P \rfloor &= \tau.\lfloor P \rfloor \\
\lfloor \epsilon_x.P \rfloor &= \tau.\lfloor P \rfloor \\
\lfloor u_x(t).P \rfloor &= u(t).\lfloor P \rfloor \\
\lfloor \bar{u}_x\langle v\rangle.P \rfloor &= \bar{u}(v).\lfloor P \rfloor \\
\lfloor P |_x Q \rfloor &= \lfloor P \rfloor | \lfloor Q \rfloor \\
\lfloor P ||_x Q \rfloor &= \lfloor P \rfloor | \lfloor Q \rfloor \\
\lfloor (\nu u)P \rfloor &= (\nu u)\lfloor P \rfloor
\end{align*}
\end{definition}

\remark The way we address the complicated meaning of the equalizer is by projecting it to $\tau.0$, which is a bit hacky, but is justified later by allowing for the reduction we want to define on the annotated calculus to project exactly on the reduction of the classic $\pi$-calculus, and because the equalizer will be associated with the axiom rule in the type system, which means it cannot be used as a prefix like $\epsilon$ or $\lambda$. That construction done now allows all else to go well, and since there is no way to clearly observe a difference in behavior between $0$ and $\tau.0$, especially since none of them interact with anything else, it is not too much of a problem to do that now, while it would create discrepancies later to not do it.

\example let's see how our first example goes with this new system. What we want is to follow the projection backwards to create an annotated term. Obviously, there will be choices to be made as to whether we want to use a parallel or a synchronization for the pre-image of the parallel in the projection, as well as choices for the $\tau$. One example of term that could work, and that we will be keeping for this paper, is the following:
\[(\nu v)(\bar{u}_x\langle v\rangle.(x\tto y) ||_y v_y(t').P) ||_x u_x(t).(\bar{t}_x\langle w\rangle.(x\tto z) ||_z Q)\]
Why we chose this form of the term is both for having a nice term for later when we introduce the reduction rules, as well as because the two name interactions and the two internal actions we saw reduce in the first example were separate, hinting at something to do with the term itself and not its potential interactions with the exterior.