%% typing system
%%%% Remove the arrows and move the annotation on the quantifier to gain
%%%% a little space
%% proof that  cut elim maintains typing

\section{Typing decorated terms with MLL}

\textcolor{red}{Section intro here, stating MLL seems a good system to type terms of the decorated calculus} %TODO

\begin{definition}
The typing system over MLL is given by the following rules:\\~\\
\indent\textbox{
	Rules for neutral elements:\\~\\
	\begin{prooftree}\infer0[NOP]{0_x \vdash x:1}\end{prooftree}\hfill
	\begin{prooftree}\hypo{P \vdash \Gamma}
		\infer1[BOT]{\epsilon_x.P \vdash \Gamma,x:\bot}\end{prooftree}\hfill
	\begin{prooftree}\infer0[SUB]{x \to y \vdash x:E^\bot,y:E}\end{prooftree}\hfill~\\~\\
	Construction rules:\\~\\
	\begin{prooftree}\hypo{P \vdash \Gamma,x:E}
		\hypo{Q \vdash \Delta,x:F}
		\infer2[PARA]{P |_x Q \vdash \Gamma,\Delta,x:E \logtensor F}\end{prooftree}\hfill
	\begin{prooftree}\hypo{P \vdash \Gamma,x:E,y:F}
		\infer1[LAM]{\lambda_xy.P \vdash \Gamma,x:E \logpar F}\end{prooftree}\hfill
	\begin{prooftree}\hypo{P \vdash \Gamma,x:E}
			\hypo{Q \vdash \Delta,x:E^\bot}
		\infer2[CUT]{P||_xQ \vdash \Gamma,\Delta}\end{prooftree}\hfill~\\~\\
	Action rules:\\~\\
	\begin{prooftree}\hypo{P \vdash \Gamma,x:A[v/t]^\bot}
		\infer1[IN]{\bar{u}_x\langle v\rangle.P \vdash \Gamma,x:\exists_u t.A^\bot}\end{prooftree}\hfill
	\begin{prooftree}\hypo{P \vdash \Gamma,x:A\;\;\;t\not\in\Gamma}
		\infer1[OUT]{u_x(t).P \vdash \Gamma,x:\forall_u t.A}\end{prooftree}\hfill~\\~\\
	Nu rule:\\~\\
	\begin{prooftree}\hypo{P \vdash \Gamma\;\;\;\; u \not\in \Gamma}
		\infer1[NU]{(\nu u)P \vdash \Gamma}\end{prooftree}
	
}
\end{definition}

One important thing to note here, is that for PARA and CUT rules, $\Gamma$ and $\Delta$ are disjoint. The other important thing is that the synchronization construction becomes a CUT operation in the typing system. This rule has elimination transformations, given by the relations $\to$, $\succ$, $\gtrsim$ and $\leadsto$ defined in \ref{reduction}. Properties we would like to emerge from such a construction would be that the arrows preserve typing, and that their application for cut elimination terminates.

\begin{proposition}
Cut elimination holds and terminates for $\to$ and $\succ$ under $\equiv$.\\
\textcolor{red}{For $\to$, we need one more hypothesis: that no quantified type remains in the type at the root of the deduction tree/all quantified types must be closed by the term or context. Should we speak about contexts and this closure condition before this property?} %TODO
\end{proposition}

\begin{myproof}
Proof by induction on the sizes of the subterms of each rule. Especially, the base cases that eliminate cut are $\epsilon$ against 0 and anything against $x \tto y$, and all other cases strictly decrease a well chosen tuple of the sizes of subterms. Make a special note that no reduction step will be blocked by an action prefix, as all action prefixes must have a complementary action to be cut against for the context to be closed on quantified variables. \textcolor{red}{Copy and adapt the proof of the temporary paper} %TODO
\end{myproof}

\remark With no exponential, it is no surprise that cut elimination terminates. We can also relax the closure hypothesis on quantified types, to the condition of proving the holding for $\succsim$ and $\leadsto$ instead, because we need the ability to commute cuts with unmatched action prefixes as well as continue cut elimination under those.\\

\textcolor{red}{Do we need a proper proposition for that last remark, as well as its proof maybe, or at least some elements for the added rules?\\~\\
Also, insert here the followed example from the other paper that is to be shown in part 1, to show here that it is not only reducing well, but also well-typed. (Question for later: Is there a term, simplest possible if exists, that reduces well without typing, but cannot be well-typed?)} %TODO