%% Not-too boring ontroduction to the subject

\section*{Introduction}

Theory of mobile processes is an important domain for research today, be it for its applications on network interactions (be it physical networks like information or energy distribution, railways...), or for its importance in mathematical logic (proof networks, concurrency and continuation transmission are some of the main points we could think of). A usual theory of mobile processes is the $\pi$-calculus, on which some work has been done and many others are still ongoing. A reference for its introduction would be \emph{The $\pi$-calculus, A Theory of Mobile Processes} by Davide Sangiorgi and David Walker~\cite{sangiorgi-2001-pi-calculus}. Especially, the $\pi$-calculus as introduced in all generality is strongly non-deterministic in the reduction of its processes. A lot of work is put into circumventing that, most of which involve developing typing systems for the $\pi$-calculus that help induce one reduction for a given typed term. Most of those works only type a fragment of the $\pi$-calculus though, and there is much work to do to try and unify these typing systems as well.\\
We will go the route of constructing a typing system as well, using linear logic because its structure is very similar to the one of the terms of the $\pi$-calculus, which is a good start to thinking such a system would work well. Preceding works in that direction include \emph{Unifying Type Systems for Mobile Processes} by Emmanuel Beffara~\cite{beffara-2015-unifying}, which is the base of this work.\\
Our goal here is to type a fragment of the $\pi$-calculus with a fragment of linear logic, with room for extending this in the future to the rest of $\pi$-calculus and linear logic. We do that by introducing a decorated form of the calculus, which annotations will allow us to follow interactions between parts of the terms to facilitate typing, by watching over one variable for each interaction. This way, we allow ourselves to embed a proof of (part of) a correct reduction path for the term, without having the difficulty of a non-deterministic behavior.\\
To make sure this work is well built, we start from classic $\pi$-calculus, and upon building its decorated variant, we also build a projection from the annotated calculus to the classic one, and build our reduction and typing systems with that projection in mind, so as to stay as faithful to the original calculus as possible.\\
Especially, we will be using the multiplicative parts of the $\pi$-calculus, and of linear logic (MLL), and we will prove that our reduction system is confluent, and that it preserves typing has a property of cut elimination. All along, we check and prove that it projects well on usual reduction of classic $\pi$-calculus, to ensure it behaves correctly in regards to this base theory.\\~\\

The first part will be focused on defining the calculi and the projection, first by reminding the part of the classic $\pi$-calculus we will build upon, then defining the decorated calculus and the projection.\\
In the second part we will define the reduction system, prove results about its confluence and show how it projects to the reduction system of the usual $\pi$-calculus.\\
The last part will introduce the typing system and show results such as type preservation for the reduction system defined before, and a way to eliminate cut by extending it.