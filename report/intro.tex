%% Not-too boring ontroduction to the subject

\section*{Introduction}

Theory of mobile processes is an important domain for research today, be it for its applications on network interactions (be it web services, the internet of things or other kinds of networks), or for its importance in logical foundations of computer science (proof nets, concurrency and continuation passing are the strong reference points in this domain). A usual theory of mobile processes is the $\pi$-calculus, on which a lot of work has been done and many open issues are still the topic of active research. A standard introduction to the topic is \emph{The $\pi$-calculus, A Theory of Mobile Processes} by Sangiorgi and Walker~\cite{sangiorgi-2001-pi-calculus}.\\

Especially, the $\pi$-calculus as introduced in all generality is strongly non-deterministic in the reduction of its processes, which is a fundamental characteristic of the systems it is conceived to model. Many approaches exist for circumventing this difficulty, most of which involve developing type systems for the $\pi$-calculus that help induce some degree of determinism. Most of these systems only work for very small fragments of the $\pi$-calculus, often excluding some of its idioms, and there is much work to do to try and unify these type systems as well, though it is not the topic of this internship.\\

We will go the route of constructing a type system as well, using linear logic because its structure is very similar to the one of the terms of the $\pi$-calculus, which is a good start to thinking such a system would work well, and we will be adding annotations on terms as well to lie down the base structure. Prior work in that direction includes \emph{On the $\pi$-Calculus and Linear Logic} by Bellin and Scott~\cite{bellin-1994-calculus}, \emph{Linear Logic Propositions as Session Types} by Caires, Pfenning and Toninho~\cite{caires-2016-linear} and \emph{Linearity and the Pi-Calculus} by Kobayashi, Pierce and Turner~\cite{kobayashi-1999-linearity} which lay some bases for this work, although this approach notably diverges from those presented in these references.\\

Our goal here is to type a fragment of the $\pi$-calculus with a fragment of linear logic, with room for extending this in the future to the rest of $\pi$-calculus and linear logic. We introduce an annotated calculus to allow for prescribing interactions, as well as facilitate the typing process, each construct being given a typing rule that revolves around the annotations it uses. The annotations allow to make the classic terms deterministic by showing at each step a reduction that would allow to continue further, and the typing part gives a guarantee that the reduction will not be blocked before the end.

To guide the rest of the work, we start from standard $\pi$-calculus, and upon building its decorated variant, we also build a projection from the annotated calculus to the classic one, and build our reduction and type systems with that projection in mind, so as to stay as faithful to the original calculus as possible.\\
Especially, we will be using the multiplicative parts of the $\pi$-calculus, and of linear logic (MLL), and we will prove that our reduction system is confluent, and that it preserves typing and has a property of cut elimination. All along, we check and prove that it projects well on usual reduction of classic $\pi$-calculus, to ensure it behaves correctly in regards to this base theory.\\

\bigskip

The first part revolves on defining the calculi and the projection, first by reminding the part of the classic $\pi$-calculus we will build upon, then defining the decorated calculus and the projection.\\
In the second part we define the reduction system, prove results about its confluence and show how it projects to the reduction system of the usual $\pi$-calculus.\\
The last part introduces the type system and show results such as type preservation for reduction and cut elimination.