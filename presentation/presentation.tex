\documentclass[11pt]{beamer}
\usetheme{Dresden}
\usepackage[utf8]{inputenc}
\usepackage[T1]{fontenc}
\usepackage[french]{babel}
\usepackage{amsthm}
\usepackage{amsmath}
\usepackage{amssymb}
\usepackage{mathrsfs}
\usepackage{setspace}
\usepackage{graphicx}
\usepackage{float}
\usepackage{multirow}
\usepackage{stmaryrd}
\usepackage{soul}
\usepackage{color}
\usepackage{verbatim}
\usepackage{alltt}
\usepackage{wrapfig}
\usepackage{pifont}
\usepackage{array}
\usepackage{colortbl}
\usepackage{ifmtarg}
\usepackage{fourier-orns}
\usepackage{pst-node}
\usepackage{pst-tree}
\usepackage{pstricks-add}
\usepackage{xspace}
\usepackage{tikz-cd}
\usepackage{pgf,tikz}
\usetikzlibrary{arrows}
\usetikzlibrary{positioning}
\usepackage{pgfplots,wrapfig}
\usepackage{ebproof}

\newcommand{\ie}{\textit{ie.}\ }
\newcommand{\NN}{\ensuremath{\mathcal{N}}\xspace}
\newcommand{\VV}{\ensuremath{\mathcal{V}}\xspace}
\newcommand{\logpar}{\ensuremath\rotatebox[origin=c]{180}{\&}}
\newcommand{\logwith}{\ensuremath\&}
\newcommand{\logplus}{\ensuremath\oplus}
\newcommand{\logtensor}{\ensuremath\otimes}
\newcommand{\tto}{\leftrightarrow}
\newcommand{\dlsucc}{\rotatebox[origin=c]{45}{$\prec$}}
\newcommand{\drsucc}{\rotatebox[origin=c]{-45}{$\succ$}}

\newtheorem{prop}{Proposition}

\author{Julien Gabet\\
sous la direction d'Emmanuel Beffara}
\title{Typage d'une version décorée du $\pi$-calcul}
%\setbeamercovered{transparent} 
\setbeamertemplate{navigation symbols}{} 
%\logo{} 
\institute{Institut de Mathématiques de Marseille} 
\date{Mars-Juin 2018\\~\\
\includegraphics[scale=0.1]{institutMathMarseille.jpg}} 
%\subject{} 
\begin{document}

\begin{frame}
\titlepage
\end{frame}


\begin{frame}
\tableofcontents
\end{frame}

\section{Introduction}

\begin{frame}
\begin{itemize}
\item Processus mobiles : objets en interaction sur un réseau, actions non déterministes.
\item Modèle habituel : $\pi$-calcul.
\item Garantir le comportement d'un objet du modèle : typage.
\item Notre approche: annoter les termes pour guider l'exécution.
\end{itemize}
\end{frame}

\section{1. Définitions}
\subsection{Le $\pi$-calcul}

\begin{frame}{Le $\pi$-calcul}
\begin{itemize}
\item Syntaxe:
\begin{align*}
P,Q::=\;\; & 0 & \text{processus inactif}\\
& P\bigl|Q & \text{mise en parallèle}\\
& u(t).P\;\;;\;\;\bar{u}v.P & \text{réception/envoi de nom}\\
& (\nu u)P & \text{privatisation de nom}
\end{align*}
\pause\item | est commutatif, associatif et admet 0 pour élément neutre.
\pause\item Règle de réduction:
\[u(t).P\bigl|\bar{u}v.Q\to P[v/t]|Q\]
Cette réduction se fait en contexte.
\end{itemize}
\end{frame}

\begin{frame}{Exemple}
Un terme simple pour voir le système de réduction:
\begin{align*}
\Bigl(u(t).P\bigl|Q\Bigr)\Bigl|\bar{u}v.R
\uncover<2-3>{\equiv& Q\Bigl|\Bigl(u(t).P\bigl|\bar{u}v.R\Bigr)\\}
\uncover<3>{\to& Q\Bigl|\Bigl(P[v/t]\bigl|R\Bigr)}
\end{align*}
\end{frame}

\begin{frame}{Exemple}
Un exemple non confluent:
\[\begin{tikzcd}[ampersand replacement=\&, column sep=tiny]
\Bigl(u(t).P\bigl|\bar{u}v.Q\Bigr)\Bigl|u(t).S \arrow[r,phantom,"\equiv"]\arrow[d] \& u(t).P\Bigl|\Bigl(\bar{u}v.Q\bigl|u(t).S\Bigr) \arrow[d] \\
\Bigl(P[v/t]\bigl|Q\Bigr)\Bigl|u(t).S \& u(t).P\Bigl|\Bigl(Q\bigl|S[v/t]\Bigr)
\end{tikzcd}\]
\end{frame}

\begin{frame}{Exemple}
Annotons l'exemple non confluent:
\[\begin{tikzcd}[ampersand replacement=\&, column sep=tiny]
\Bigl(u_x(t).P\bigl|\bigl|_x\bar{u}_xv.Q\Bigr)\Bigl|\Bigl|_yu_y(t).S \arrow[r,phantom,"\not\equiv"]\arrow[d] \& u_x(t).P\Bigl|\Bigl|_x\Bigl(\bar{u}_yv.Q\bigl|\bigl|_yu_y(t).S\Bigr) \arrow[d] \\
\Bigl(P[v/t]\bigl|\bigl|_xQ\Bigr)\Bigl|\Bigl|_yu_y(t).S \& u_x(t).P\Bigl|\Bigl|_x\Bigl(Q\bigl|\bigl|_yS[v/t]\Bigr)
\end{tikzcd}\]
\end{frame}

\subsection{Une version annotée du $\pi$-calcul}

\begin{frame}{Les termes annotés}
Les termes annotés sont définis comme suit:
\begin{align*}
P,Q::= & x\tto y\;\;;\;\;0_x\;\;; & \text{termes de base}\\
& P\bigl|_xQ\;\;;\;\;P\bigl|\bigl|_xQ\;\;; & \text{parallèle et synchronisation}\\
& u_x(t).P\;\;;\;\;\bar{u}_xv.P\;\;; & \text{préfixes d'action}\\
& \epsilon_x.P\;\;;\;\;\lambda_xy.P\;\;; & \text{préfixes de planification}\\
& (\nu u)P & \text{privatisation de nom}
\end{align*}
où $x\in fv(P)\cap fv(Q)$ pour les règles | et ||,\\
$x\not\in fv(P)$ pour la règle $\epsilon$, et $x,y\in fv(P)$ pour les autres règles.
\end{frame}

\section{2. Exécution du calcul annoté}

\begin{frame}{La réduction}
\begin{itemize}
\item Règles de réduction:
\begin{align*}
\epsilon_x.P ||_x 0_x &\to P & (P|_xQ)||_x\lambda_xy.R &\to P||_x(Q[y/x]||_yR)\\
P||_x x\tto y &\to P[y/x] & \bar{u}_xv.P||_xu_x(t).Q &\to P||_x Q[v/t]
\end{align*}
\pause\item Règles de remontée du ||:\\
Si le contexte l'exige, on note $P^y$ si le terme $P$ contient $y$.\\
$(P^y|_xQ)||_yR \succ (P||_yR)|_xQ$\hfill$(P|_xQ^y)||_yR \succ P|_x(Q||_yR)$
\end{itemize}
\end{frame}

\begin{frame}{La réduction}
Il faut une congruence pour faire confluer $\succ$:
\begin{align*}
(P^y|_xQ)|_yR &\equiv (P^x|_yR)|_xQ & P|_x(Q|_yR^x) &\equiv Q|_y(P|_xR^y)\\
(P|_xQ^y)|_yR &\equiv P|_x(Q^x|_yR)
\end{align*}
Ces règles font commuter les relations selon des variables distinctes.
\begin{prop}
\begin{itemize}
\item $\to$ est fortement confluent
\item $\succ$ est confluent à $\equiv$ près.
\end{itemize}
\end{prop}
\end{frame}

\begin{frame}{Confluence de $\to$ et $\succ$}
Pour $\to$, seule la règle $x \tto y$ interagit avec elle-même:
\[\begin{tikzcd}[ampersand replacement=\&, column sep=tiny]
\& x\tto y ||_x x\tto z \arrow[dl] \arrow[dr] \&\\
(x\tto z)[y/x]\arrow[d, phantom, "\shortparallel"] \& \& (x\tto y)[z/x]\arrow[d, phantom, "\shortparallel"]\\
y\tto z \& = \& z\tto y
\end{tikzcd}\]
\end{frame}

\begin{frame}{Confluence de $\to$ et $\succ$}
Pour $\succ$, on regarde un cas (les autres sont similaires):
\begin{tikzcd}[ampersand replacement=\&, column sep=tiny]
\& (P^y|_xQ)||_y(R^y|_zS)\arrow[dl,phantom,"\dlsucc_{left}"xshift=2.5ex]\arrow[dr,phantom,"\drsucc_{right}"xshift=2.5ex] \&\\
\Bigl(P^y||_y(R^y|_zS)\Bigr)\Bigl|_xQ\arrow[d,phantom,"\curlyvee"] \& \& \Bigl((P^y|_xQ)||_yR^y\Bigr)\Bigl|_zS\arrow[d,phantom,"\curlyvee"]\\
\Bigl((P^y||_yR^y)|_zS\Bigr)\Bigl|_xQ \& \equiv \& \Bigl((P^y||_yR^y)|_xQ\Bigr)\Bigl|_zS
\end{tikzcd}
\end{frame}

\begin{frame}{Des termes dont la réduction se bloque}
\begin{itemize}
\item une action contre un autre préfixe :
\[\epsilon_x.0_y||_xu_x(t).0_x\]
\item deux actions qui s'entremêlent :
\[\bar{u}_xa.v_x(t).P||_x\bar{v}_xb.u_x(t').Q\]
\end{itemize}
\end{frame}

\section{3. Typage du calcul décoré avec MLL}
\subsection{Le système de types}

\begin{frame}{Les règles de typage}
\small{
\indent\hfill\begin{prooftree}\infer0[nop]{0_x\vdash x:1}\end{prooftree}\hfill
\begin{prooftree}\hypo{P\vdash\Gamma}\hypo{x\not\in\Gamma}\infer2[bot]{\epsilon_x.P\vdash\Gamma,x:\bot}\end{prooftree}\hfill~\\~\\~\\
\indent\hfill\begin{prooftree}\infer0[ax]{x\tto y\vdash x:E^\bot,y:E}\end{prooftree}\hfill
\begin{prooftree}\hypo{P\vdash\Gamma,x:E}\hypo{Q\vdash\Delta,x:E^\bot}\infer2[cut]{P||_xQ\vdash\Gamma,\Delta}\end{prooftree}\hfill~\\~\\~\\
\begin{prooftree}\hypo{P\vdash\Gamma,x:E}\hypo{Q\vdash\Delta,x:F}\infer2[para]{P|_xQ\vdash\Gamma,\Delta,x:E\logtensor F}\end{prooftree}\hfill
\begin{prooftree}\hypo{P\vdash\Gamma,x:E,y:F}\infer1[lam]{\lambda_xy.P\vdash\Gamma,x:E\logpar F}\end{prooftree}\\~\\~\\
\indent\hfill\begin{prooftree}\hypo{P\vdash\Gamma,x:A[v/t]^\bot}\infer1[in]{\bar{u}_xv.P\vdash\Gamma,x:\exists_ut.A^\bot}\end{prooftree}\hfill
\begin{prooftree}\hypo{P\vdash\Gamma,x:A}\hypo{t\not\in\Gamma}\infer2[out]{u_x(t).P\vdash\Gamma,x:\forall_ut.A}\end{prooftree}\hfill~}
\end{frame}

\begin{frame}{Propriétés des relations typées}
\begin{prop}Les relations $\to$, $\succ$ et $\equiv$ préservent le type,\\
\ie si $P\vdash\Gamma$ et $P\alpha P'$, alors $P'\vdash\Gamma$.\hfill(pour $\alpha$ l'une de ces relations)\end{prop}
$\bullet$ La preuve est standard, on regarde chaque cas de réécriture avec les règles de typage correspondantes.
\pause\begin{prop}Les relations $\to$ et $\succ$ terminent, \ie n'admettent pas de chaîne de réécriture infinie.\end{prop}
$\bullet$ La preuve est également assez standard, pour un bon choix de mesure de terminaison. Une mesure qui fonctionne est la somme des tailles des sous-termes immédiats de la coupure/synchronisation.
\end{frame}

\subsection{L'élimination de la coupure}

\begin{frame}{Élimination de la coupure}
On ajoute des règles pour passer et agir sous les préfixes:
\begin{align*}
\epsilon_x.P||_yQ &\succsim \epsilon_x.(P||_yQ) & u_x(t).P||_yQ &\succsim u_x(t).(P||_yQ)\\
\lambda_xy.P||_zQ &\succsim \lambda_xy.(P||_zQ) & \bar{u}_xv.P||_yQ &\succsim \bar{u}_xv.(P||_yQ)
\end{align*}
\begin{align*}
P|_x \alpha_y.Q &\cong \alpha_y.(P|_xQ) & \alpha\cdot \in \{\epsilon_\cdot,\lambda_\cdot z,u_\cdot(t),\bar{u}_\cdot v\}, y,t\not\in fv(P)\\
\alpha_x.\beta_y.P &\cong \beta_y.\alpha_x.P & \alpha_\cdot,\beta_\cdot \in \{\epsilon_\cdot,\lambda_\cdot z,u_\cdot(t),\bar{u}_\cdot v\},x\neq y
\end{align*}
Et $\to$ est étendue en $\leadsto$ qui agit sous les préfixes.\\
\pause$\bullet$ Ces nouvelles règles préservent le type, $\leadsto$ et $\succsim$ terminent et confluent comme avant.\\
\pause$\bullet$ Ces règles éliminent la coupure.
\end{frame}

\section{Conclusion}

\begin{frame}{Conclusion}
Les aspects techniques non abordés:
\begin{itemize}
\item effacement des annotations,
\item actions internes (non observable depuis l'extérieur),
\item la confluence de $\to$ après quotient par $\equiv$,
\item les limitations techniques de cette approche.
\end{itemize}
Les aspects à rechercher encore:
\begin{itemize}
\item partie additive de $\pi$ et de LL,
\item réplication de $\pi$ et modalités de LL,
\item annoter un terme à partir d'une réduction connue ?
\item place du système annoté parmi les systèmes de types existants
\end{itemize}
\end{frame}

\end{document}