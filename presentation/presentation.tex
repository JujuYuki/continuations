\documentclass[11pt]{beamer}
\usetheme{Dresden}
\usepackage[utf8]{inputenc}
\usepackage[T1]{fontenc}
\usepackage[french]{babel}
\usepackage{amsthm}
\usepackage{amsmath}
\usepackage{amssymb}
\usepackage{mathrsfs}
\usepackage{setspace}
\usepackage{graphicx}
\usepackage{float}
\usepackage{multirow}
\usepackage{stmaryrd}
\usepackage{soul}
\usepackage{color}
\usepackage{verbatim}
\usepackage{alltt}
\usepackage{wrapfig}
\usepackage{pifont}
\usepackage{array}
\usepackage{colortbl}
\usepackage{ifmtarg}
\usepackage{fourier-orns}
\usepackage{pst-node}
\usepackage{pst-tree}
\usepackage{pstricks-add}
\usepackage{xspace}
\usepackage{tikz-cd}
\usepackage{pgf,tikz}
\usetikzlibrary{arrows}
\usetikzlibrary{positioning}
\usepackage{pgfplots,wrapfig}
\usepackage{ebproof}

\newcommand{\ie}{\textit{ie.}\ }
\newcommand{\NN}{\ensuremath{\mathcal{N}}\xspace}
\newcommand{\VV}{\ensuremath{\mathcal{V}}\xspace}
\newcommand{\logpar}{\ensuremath\rotatebox[origin=c]{180}{\&}}
\newcommand{\logwith}{\ensuremath\&}
\newcommand{\logplus}{\ensuremath\oplus}
\newcommand{\logtensor}{\ensuremath\otimes}
\newcommand{\tto}{\leftrightarrow}

\author{Julien Gabet\\
sous la direction d'Emmanuel Beffara}
\title{Typage d'une version décorée du $\pi$-calcul}
%\setbeamercovered{transparent} 
\setbeamertemplate{navigation symbols}{} 
%\logo{} 
\institute{Institut de Mathématiques de Marseille} 
\date{Mars-Juin 2018\\~\\
\includegraphics[scale=0.1]{institutMathMarseille.jpg}} 
%\subject{} 
\begin{document}

\begin{frame}
\titlepage
\end{frame}


\begin{frame}
\tableofcontents
\end{frame}

\section{1. Les deux systèmes de calcul}
\subsection{Le $\pi$-calcul}

\begin{frame}{$\pi$-termes}
\begin{definition}
Soit $\NN:=\{t,u,v,\ldots\}$ un ensemble dénombrable de noms. Les termes du $\pi$-calcul sont donnés par la grammaire suivante:
\[P,Q::=0\;\;;\;\;P|Q\;\;;\;\;u(t).P\;\;;\;\;\bar{u}(v).P\;\;;\;\;\tau.P\;\;;\;\;(\nu u)P\]
et l'ensemble des noms libres d'un terme $fn(P)$ par induction sur les termes comme suit:
\begin{align*}
fn(0)&=\emptyset & fn(P|Q)&=fn(P)\cup fn(Q) \\
fn(u(t).P)&=(fn(P)\backslash\{t\})\cup\{u\} & fn(\bar{u}(v).P)&=fn(P)\cup\{u,v\} \\
fn(\tau.P)&=fn(P) & fn((\nu u)P)&=fn(P)\backslash\{u\}
\end{align*}
\end{definition}
\end{frame}

\begin{frame}{Congruence structurelle et réduction}
\begin{definition}
On définit la congruence structurelle comme une congruence sur les termes avec les axiomes suivants:
\begin{align*}
P|0&\equiv P & P|Q&\equiv Q|P\\
P|(Q|R)&\equiv (P|Q)|R & (\nu u)(\nu v)P&\equiv(\nu v)(\nu u)P\\
(\nu u)0&\equiv 0 & (\nu u)(P|Q)&\equiv P|(\nu u)Q\;\;\text{ if }u\not\in fn(P)
\end{align*}
et on se donne les règles de réduction suivantes:\\
\indent\hfill\begin{prooftree}\infer0{u(t).P|\bar{u}(v).Q\to P[v/t]|Q}\end{prooftree}\hfill
\begin{prooftree}\infer0{\tau.P\to P}\end{prooftree}\hfill~\\~\\
\begin{prooftree}\hypo{P\to P'}\infer1{P|Q\to P'|Q}\end{prooftree}\hfill
\begin{prooftree}\hypo{P\to P'}\infer1{(\nu u)P\to(\nu u)P'}\end{prooftree}\hfill
\begin{prooftree}\hypo{P\equiv Q\to Q'\equiv P'}\infer1{P\to P'}\end{prooftree}
\end{definition}
\end{frame}

\begin{frame}{Exemple}
Un terme simple pour voir le système de réduction:
\begin{align*}
\Bigl(u(t).P|Q\Bigr)\Bigl|\bar{u}(v).R
\uncover<2-4>{\equiv& \Bigl(Q|u(t).P\Bigr)\Bigl|\bar{u}(v).R\\}
\uncover<3-4>{\equiv& Q\Bigl|\Bigl(u(t).P|\bar{u}(v).R\Bigr)\\}
\uncover<4>{\to& Q\Bigl|\Bigl(P[v/t]|R\Bigr)}
\end{align*}
\end{frame}

\begin{frame}{Exemple}
Un exemple non confluent:
\[\begin{tikzcd}[ampersand replacement=\&, column sep=tiny]
(u(t).P|\bar{u}(v).Q)|u(t).S \arrow[r,phantom,"\equiv"]\arrow[d] \& u(t).P|(\bar{u}(v).Q|u(t).S) \arrow[d] \\
(P[v/t]|Q)|u(t).S \& u(t).P|(Q|S[v/t])
\end{tikzcd}\]
\end{frame}

\subsection{Une version annotée du $\pi$-calcul}

\begin{frame}{Les termes annotés}
\begin{definition}
Soient $\NN:=\{t,u,v,\ldots\}$ et $\VV:=\{x,y,z,\ldots\}$ deux ensembles de noms et de variables, dénombrables et disjoints. Les termes annotés sont donnés par la grammaire suivante:
\begin{align*}
P,Q::= & x\tto y\;\;;\;\;0_x\;\;; & \text{termes de base}\\
& P|_xQ\;\;;\;\;P||_xQ\;\;; & \text{parallèle et synchronisation}\\
& u_x(t).P\;\;;\;\;\bar{u}_x(v).P\;\;; & \text{préfixes d'action}\\
& \epsilon_x.P\;\;;\;\;\lambda_xy.P\;\;; & \text{préfixes de planification}\\
& (\nu u)P & \text{privatisation de nom}
\end{align*}
où $x\in fv(P)\cap fv(Q)$ pour les règles | et ||,\\
$x\not\in fv(P)$ pour la règle $\epsilon$, et $x,y\in fv(P)$ pour les autres règles.
\end{definition}
\end{frame}

\begin{frame}{Les termes annotés}
\begin{definition}
Et où l'ensemble des variables libres d'un terme $fv(P)$ est défini par induction sur les termes:
\begin{align*}
fv(x\tto y) &= \{x,y\} & fv(0_x) &= \{x\}\\
fv(P|_xQ) &= fv(P)\cup fv(Q) & fv(P||_xQ) &= (fv(P)\cup fv(Q))\backslash\{x\}\\
fv(u_x(t).P) &= fv(P) & fv(\bar{u}_x(v).P) &= fv(P) \\
fv(\epsilon_x.P) &= fv(P)\cup\{x\} & fv(\lambda_xy.P) &= fv(P)\backslash\{y\}\\
fv((\nu u)P) &= fv(P)
\end{align*}
L'ensemble des noms libres est défini comme pour le $\pi$-calcul classique en considérant $x\tto y$ comme 0 et $\lambda,\epsilon$ comme $\tau$
\end{definition}
\end{frame}

\begin{frame}{Projection des termes annotés sur $\pi$}
\begin{definition}
On définit la projection comme suit:
\begin{align*}
\lfloor 0_x \rfloor &= 0 & \lfloor x\tto y \rfloor &= \tau.0\\
\lfloor P|_xQ \rfloor &= \lfloor P \rfloor|\lfloor Q\rfloor & \lfloor P||_xQ \rfloor &= \lfloor P \rfloor|\lfloor Q\rfloor \\
\lfloor u_x(t).P \rfloor &= u(t).\lfloor P\rfloor & \lfloor \bar{u}_x(v).P \rfloor &= \bar{u}(v).\lfloor P\rfloor\\
\lfloor \epsilon_x.P \rfloor &= \tau.\lfloor P\rfloor & \lfloor \lambda_xy.P \rfloor &= \tau.\lfloor P\rfloor\\
\lfloor(\nu u)P\rfloor &= (\nu u)\lfloor P\rfloor
\end{align*}
\end{definition}
\end{frame}

%ajouter exemple ici p-ê %TODO

\section{2.Exécution du calcul annoté}
\subsection{Réduction}

\begin{frame}{La réduction}
Ici la flèche $\to$ et la remontée $\succ$ %TODO
\end{frame}

\begin{frame}{La réduction}
Ici les règles de congruence $\equiv$ %TODO
\end{frame}

\begin{frame}{Confluence de $\to$ et $\succ$}
Prop: confluence de $\to$ %TODO
\end{frame}

\begin{frame}{Confluence de $\to$ et $\succ$}
Prop: confluence de $\succ$ à $\equiv$ près %TODO
\end{frame}

\begin{frame}{Confluence de $\to$ et $\succ$}
Fermeture spécifique de $\equiv$ pour $\to$, avec $\succ$: donner un exemple %TODO
\end{frame}

%Quid de la projection? Partie spécifique? ou au tableau pendant les slides précédents? %TODO

\section{3. Typage du calcul décoré avec MLL}

\begin{frame}
Le système de types: langage %TODO
\end{frame}

\begin{frame}
Le système de types: règles %TODO
\end{frame}

\begin{frame}
Préservation du type: proposition %TODO
\end{frame}

\begin{frame}
Préservation du type: un exemple? %TODO
\end{frame}

\begin{frame}
Terminaison de $\to$ et $\succ$ + un exemple peut-être %TODO
\end{frame}

\begin{frame}
Extension du système de réduction %TODO
\end{frame}

\begin{frame}
Élimination de la coupure %TODO
\end{frame}

\end{document}