\documentclass[11pt]{beamer}
\usetheme{Dresden}
\usepackage[utf8]{inputenc}
\usepackage[T1]{fontenc}
\usepackage[french]{babel}
\usepackage{amsthm}
\usepackage{amsmath}
\usepackage{amssymb}
\usepackage{mathrsfs}
\usepackage{setspace}
\usepackage{graphicx}
\usepackage{float}
\usepackage{multirow}
\usepackage{stmaryrd}
\usepackage{soul}
\usepackage{color}
\usepackage{verbatim}
\usepackage{alltt}
\usepackage{wrapfig}
\usepackage{pifont}
\usepackage{array}
\usepackage{colortbl}
\usepackage{ifmtarg}
\usepackage{fourier-orns}
\usepackage{pst-node}
\usepackage{pst-tree}
\usepackage{pstricks-add}
\usepackage{xspace}
\usepackage{tikz-cd}
\usepackage{pgf,tikz}
\usetikzlibrary{arrows}
\usetikzlibrary{positioning}
\usepackage{pgfplots,wrapfig}
\usepackage{ebproof}

\newcommand{\ie}{\textit{ie.}\ }
\newcommand{\NN}{\ensuremath{\mathcal{N}}\xspace}
\newcommand{\VV}{\ensuremath{\mathcal{V}}\xspace}
\newcommand{\logpar}{\ensuremath\rotatebox[origin=c]{180}{\&}}
\newcommand{\logwith}{\ensuremath\&}
\newcommand{\logplus}{\ensuremath\oplus}
\newcommand{\logtensor}{\ensuremath\otimes}
\newcommand{\tto}{\leftrightarrow}
\newcommand{\dlsucc}{\rotatebox[origin=c]{45}{$\prec$}}
\newcommand{\drsucc}{\rotatebox[origin=c]{-45}{$\succ$}}

\newtheorem{prop}{Proposition}

\author{Julien Gabet\\
sous la direction d'Emmanuel Beffara}
\title{Typage d'une version décorée du $\pi$-calcul}
%\setbeamercovered{transparent} 
\setbeamertemplate{navigation symbols}{} 
%\logo{} 
\institute{Institut de Mathématiques de Marseille} 
\date{Mars-Juin 2018\\~\\
\includegraphics[scale=0.1]{institutMathMarseille.jpg}} 
%\subject{} 
\begin{document}

\begin{frame}
\titlepage
\end{frame}


\begin{frame}
\tableofcontents
\end{frame}

\section{Introduction}

\begin{frame}
\begin{itemize}
\item Processus mobiles : objets en interaction sur un réseau, actions non déterministes.
\item Modèle habituel : $\pi$-calcul.
\item Vérifier le comportement d'un objet du modèle : typage.
\item Notre approche: annoter les termes pour guider l'exécution.
\end{itemize}
\end{frame}

\section{1. Définitions}
\subsection{Le $\pi$-calcul}

\begin{frame}{Le $\pi$-calcul}
\begin{itemize}
\item $P,Q::=0\;\;;\;\;P|Q\;\;;\;\;u(t).P\;\;;\;\;\bar{u}(v).P\;\;;\;\;\tau.P\;\;;\;\;(\nu u)P$
\pause\item Congruence structurelle:
\begin{align*}
P|0&\equiv P & P|Q&\equiv Q|P\\
P|(Q|R)&\equiv (P|Q)|R & (\nu u)(\nu v)P&\equiv(\nu v)(\nu u)P\\
(\nu u)0&\equiv 0 & (\nu u)(P|Q)&\equiv P|(\nu u)Q\;\;\text{ if }u\not\in fn(P)
\end{align*}
\pause\item Règles de réduction:\\
\indent\hfill\begin{prooftree}\infer0{u(t).P|\bar{u}(v).Q\to P[v/t]|Q}\end{prooftree}\hfill
\begin{prooftree}\infer0{\tau.P\to P}\end{prooftree}\hfill~\\~\\
\begin{prooftree}\hypo{P\to P'}\infer1{P|Q\to P'|Q}\end{prooftree}\hfill
\begin{prooftree}\hypo{P\to P'}\infer1{(\nu u)P\to(\nu u)P'}\end{prooftree}\hfill
\begin{prooftree}\hypo{P\equiv Q\to Q'\equiv P'}\infer1{P\to P'}\end{prooftree}
\end{itemize}
\end{frame}

\begin{frame}{Exemple}
Un terme simple pour voir le système de réduction:
\begin{align*}
\Bigl(u(t).P|Q\Bigr)\Bigl|\bar{u}(v).R
\uncover<2-4>{\equiv& \Bigl(Q|u(t).P\Bigr)\Bigl|\bar{u}(v).R\\}
\uncover<3-4>{\equiv& Q\Bigl|\Bigl(u(t).P|\bar{u}(v).R\Bigr)\\}
\uncover<4>{\to& Q\Bigl|\Bigl(P[v/t]|R\Bigr)}
\end{align*}
\end{frame}

\begin{frame}{Exemple}
Un exemple non confluent:
\[\begin{tikzcd}[ampersand replacement=\&, column sep=tiny]
\Bigl(u(t).P|\bar{u}(v).Q\Bigr)\Bigl|u(t).S \arrow[r,phantom,"\equiv"]\arrow[d] \& u(t).P\Bigl|\Bigl(\bar{u}(v).Q|u(t).S\Bigr) \arrow[d] \\
\Bigl(P[v/t]|Q\Bigr)\Bigl|u(t).S \& u(t).P\Bigl|\Bigl(Q|S[v/t]\Bigr)
\end{tikzcd}\]
\end{frame}

\subsection{Une version annotée du $\pi$-calcul}

\begin{frame}{Les termes annotés}
Les termes annotés sont définis comme suit:
\begin{align*}
P,Q::= & x\tto y\;\;;\;\;0_x\;\;; & \text{termes de base}\\
& P|_xQ\;\;;\;\;P||_xQ\;\;; & \text{parallèle et synchronisation}\\
& u_x(t).P\;\;;\;\;\bar{u}_x(v).P\;\;; & \text{préfixes d'action}\\
& \epsilon_x.P\;\;;\;\;\lambda_xy.P\;\;; & \text{préfixes de planification}\\
& (\nu u)P & \text{privatisation de nom}
\end{align*}
où $x\in fv(P)\cap fv(Q)$ pour les règles | et ||,\\
$x\not\in fv(P)$ pour la règle $\epsilon$, et $x,y\in fv(P)$ pour les autres règles.
\end{frame}

\begin{frame}{Les termes annotés}
Et où l'ensemble des variables libres d'un terme $fv(P)$ est défini par induction sur les termes:
\begin{align*}
fv(x\tto y) &= \{x,y\} & fv(0_x) &= \{x\}\\
fv(P|_xQ) &= fv(P)\cup fv(Q) & fv(P||_xQ) &= (fv(P)\cup fv(Q))\backslash\{x\}\\
fv(u_x(t).P) &= fv(P) & fv(\bar{u}_x(v).P) &= fv(P) \\
fv(\epsilon_x.P) &= fv(P)\cup\{x\} & fv(\lambda_xy.P) &= fv(P)\backslash\{y\}\\
fv((\nu u)P) &= fv(P)
\end{align*}
\end{frame}

\section{2. Exécution du calcul annoté}

\begin{frame}{La réduction}
\begin{itemize}
\item Règles de réduction:
\begin{align*}
\epsilon_x.P ||_x 0_x &\to P & (P|_xQ)||_x\lambda_xy.R &\to P||_x(Q[y/x]||_yR)\\
P||_x x\tto y &\to P[y/x] & \lambda_xy.R||_x(P|_xQ) &\to (R||_xP)||_yQ[y/x]\\
& & \bar{u}_x(v).P||_xu_x(t).Q &\to P||_x Q[v/t]
\end{align*}
\pause\item Règles de remontée du ||:\\
Si le contexte l'exige, on note $P^y$ si le terme $P$ contient $y$.\\
$(P^y|_xQ)||_yR \succ (P||_yR)|_xQ$\hfill$(P|_xQ^y)||_yR \succ P|_x(Q||_yR)$\\
$(\nu u)P||_xQ \succ (\nu u)(P||_xQ)$\hfill si $u\not\in fn(Q)$
\end{itemize}
\end{frame}

\begin{frame}{La réduction}
Il faut une congruence pour faire confluer $\succ$:
\begin{align*}
(P^y|_xQ)|_yR &\equiv (P^x|_yR)|_xQ & (\nu u)(\nu v)P &\equiv (\nu v)(\nu u)P\\
P|_x(Q|_yR^x) &\equiv Q|_y(P|_xR^y) & (\nu u)0_x &\equiv 0_x\\
(P|_xQ^y)|_yR &\equiv P|_x(Q^x|_yR) & (\nu u)P|_xQ &\equiv (\nu u)(P|_xQ) & u\not\in Q
\end{align*}
\begin{prop}
\begin{itemize}
\item $\to$ est fortement confluent
\item $\succ$ est confluent à $\equiv$ près.
\end{itemize}
\end{prop}
\end{frame}

\begin{frame}{Confluence de $\to$ et $\succ$}
Pour $\to$, seule la règle $x \tto y$ interagit avec elle-même:
\[\begin{tikzcd}[ampersand replacement=\&, column sep=tiny]
\& x\tto y ||_x x\tto z \arrow[dl] \arrow[dr] \&\\
(x\tto z)[y/x]\arrow[d, phantom, "\shortparallel"] \& \& (x\tto y)[z/x]\arrow[d, phantom, "\shortparallel"]\\
y\tto z \& = \& z\tto y
\end{tikzcd}\]
\end{frame}

\begin{frame}{Confluence de $\to$ et $\succ$}
Pour $\succ$, on regarde un cas (les autres sont similaires):
\begin{tikzcd}[ampersand replacement=\&, column sep=tiny]
\& (P^y|_xQ)||_y(R^y|_zS)\arrow[dl,phantom,"\dlsucc_{left}"xshift=2.5ex]\arrow[dr,phantom,"\drsucc_{right}"xshift=2.5ex] \&\\
\Bigl(P^y||_y(R^y|_zS)\Bigr)\Bigl|_xQ\arrow[d,phantom,"\curlyvee"] \& \& \Bigl((P^y|_xQ)||_yR^y\Bigr)\Bigl|_zS\arrow[d,phantom,"\curlyvee"]\\
\Bigl((P^y||_yR^y)|_zS\Bigr)\Bigl|_xQ \& \equiv \& \Bigl((P^y||_yR^y)|_xQ\Bigr)\Bigl|_zS
\end{tikzcd}
\end{frame}

\begin{frame}{Confluence de $\to$ et $\succ$}
Fermeture spécifique de $\equiv$ pour $\to$, avec $\succ$: donner un exemple %TODO
\end{frame}

\section{3. Typage du calcul décoré avec MLL}
\subsection{Le système de types}

\begin{frame}
Le système de types: langage\\ %TODO
et règles %TODO
\end{frame}

\begin{frame}
Préservation du type: proposition et un exemple (en 2 slides ?) %TODO
\end{frame}

\begin{frame}
Terminaison de $\to$ et $\succ$ (+ un exemple ?) %TODO
\end{frame}

\subsection{L'élimination des coupures}

\begin{frame}
Extension du système de réduction %TODO
\end{frame}

\begin{frame}
Élimination des coupures %TODO
\end{frame}

\section{Conclusion}

\begin{frame}{Conclusion}
conclure ici: projection (on ne la présente pas vraiment ici, juste évoquée à l'oral éventuellement), limitations des choix faits.
\end{frame}

\end{document}